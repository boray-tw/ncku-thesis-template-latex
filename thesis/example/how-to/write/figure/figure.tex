% ------------------------------------------------
\StartSection{圖片 Figure}{chapter:how-to:write:figure}
% ------------------------------------------------

\StartSubSection{圖片小知識}

圖片為兩類別: 點陣圖或向量圖 (參考 Fig \RefTo{fig:how-to:write:figure:format}) .\\

  \InsertFigure
    [scale=0.3,
      caption={圖片類別},
      label={fig:how-to:write:figure:format}]
    {./example/how-to/write/figure/pic/graph-format.png}

點陣圖(如.jpg, .gif,.png, .tiff)在相機和網頁中十分常見, 優點是幾乎能應用在不同地方/工具, 缺點在放大縮小時會出現失真的情況, 所以為了更清晰, 則要更高的解析度的圖片, 這時候就會大大增加圖片大小, 而這大小會直接影響所產生出來論文PDF的大小.

向量圖(如.pdf, .eps, .svg)在學術界內用來放在論文中是非常常見. 優點是不會因放大縮小而造成內容變型, 所以是十分有用的. 缺點是必須使用一些特定的工具才能顯示或產生出來. 而LaTex主要使用這一類圖檔. 但LaTex對SVG的支援十分不好, 故這模板沒無提供插入SVG檔, 故強烈推薦使用PDF為主要的格式.

PDF可同時為點陣圖或向量圖, 主要是看你提供什麼圖片格式來轉成PDF. 另一種Windows增加型中繼檔(.emf)都同樣是點陣圖或向量圖, 而由於Word是沒法直接插入PDF/EPS/SVG的圖檔, 所以需先轉成這格式, 以保留向量圖的品質.

% ------------------------------------------------
\newpage
\StartSubSection{轉換格式}

以下為一些已知的方式可把圖像轉成向量圖格式, 按鈕的位置有可能不一樣, 但應該都會在那些地方中. (以下使用工具版本為: Adobe Acrobat XI Pro, Adobe Illustor CS5, Adobe Photoshop CS5, Microsoft Visio Professional 2013, Microsoft Office Professional 2013 [Excel/PowerPoint])

\begin{description}
  \item[SVG - > PDF] \hfill\\
    如果你有安裝學校的Adobe Acrobat Pro (沒有的話都推薦你安裝. 因為交給圖書館的電子檔時, 你起碼要對PDF檔上鎖), 直接對那個SVG檔右鍵, 就會有`轉換成Adobe PDF' 這個選項.

  \item[Adobe Illustor - > EMF] \hfill\\
    `檔案' -> `轉存' -> 格式拉到最下面就有了.

  \item[Adobe Illustor - > PDF] \hfill\\
    `檔案' -> `另存新檔' -> 在格式中間位置 -> 如果你的是成品的話, 則可考慮把 `保留Illustrator編輯能力', `內嵌頁面縮圖'都拿掉以減少PDF檔的大小.

  \item[Adobe Illustor - > SVG] \hfill\\
    `檔案' -> `另存新檔' -> 在格式最底的位置 -> 如果你想不到有什麼設定, 直接按`確定'就好了. 同時都推薦在`影像'的選項設定為`嵌入'以去掉任何影像有位置不對而造成SVG檔有什麼問題.

  \item[Adobe Photoshop - > PDF] \hfill\\
    `檔案' -> `另存新檔' -> 在格式中下的位置 -> 如果你的是成品的話, 則可考慮把 `保留Photoshop編輯能力'拿掉以減少PDF檔的大小.

  \item[Visio - > PDF] \hfill\\
    `檔案' -> `匯出' -> `建立PDF/XPS'.

  \item[Visio - > SVG] \hfill\\
    `檔案' -> `匯出' -> `變更檔案類型' -> `SVG可縮放向量圖形' -> 下面的`另存新檔'.

  \item[Visio - > EMF] \hfill\\
    `檔案' -> `匯出' -> `變更檔案類型' -> `EMF 增加型中繼檔' -> 下面的`另存新檔'.

  \newpage
  \item[PowerPoint - > EMF] \hfill
    \begin{enumerate}
      \item `檔案' -> `匯出' -> `變更檔案類型' -> `儲存成其他檔案類型'
      \item 下面的`另存新檔' -> 在格式最底部位置選擇`Windows增加型中繼檔' -> `僅此投影片'即可.
      \item 當然都可選擇`所有投影片', 只是`僅此投影片'即可儲存跟PowerPoint同檔名的EMF檔. 而`所有投影片'會把多張的EMF檔放在一個資料夾中.
    \end{enumerate}

  \item[Excel - > PDF] Excel可以把所做出來的圖表轉成PDF內容.
    \begin{enumerate}
      \item 在Excel中對某個圖表按一下左鍵, 之後 `檔案' -> `匯出' -> `建立PDF/XPS'.
      \item 所做出來的PDF應該是一張A4, 所以要做裁切. 打開那個PDF, 右上方有一個`工具' -> 右邊多了一個工具列.
      \item 按`裁切', 之後在圖中的任何地方按2下左鍵, 把`移除白色邊距'打勾 (如右圖中沒任何反應, 重新打勾一下看看), 之後按`確定'即可.
      \item 按`Ctrl + S'來儲存即可.
    \end{enumerate}

  \item[Excel - > EMF] Excel沒有任何直接方式把圖表轉成EMF, 但有方法來間接轉換.
    \begin{enumerate}
      \item 先做一次\textbf{Excel - > PDF}的方法.
      \item `檔案' -> `另存新檔' -> 在格式中間位置選擇`PowerPoint簡報(*.pptx)'.
      \item 之後做一次\textbf{PowerPoint - > EMF}的方法即可.
    \end{enumerate}
\end{description}

\noindent \textbf{注意:} 一些系所可能會使用自己的程式或一些工具以製造出SVG檔, 但以我經驗發現有時候有些工具所產出的SVG檔並不能在日常的工具或瀏覽器正常顯示 (應該是SVG中的XML有某程度的內容跟公認的內容不太一樣所造成的). 所以我會推薦先檢查這SVG檔是否在手上的工具或程式都顯示正常, 之後再把這SVG轉成PDF, 以方便匯入到論文之中, 同時能保證這SVG中的內容沒有出現任何變型或錯誤.

% ------------------------------------------------
\newpage
\StartSubSection{使用介紹}

插入圖片其實有很多的玩法, 但是在畢業論文中, 它的放置位置則是非常固定的, 都是以中間為主, 並插入單/多張圖片. 因為圖片位置都是固定的, 所以本模板針對了插入單張或多張來設計, 之後的章節會對這2個方式的使用作詳細說明.

要注意的是, 圖片在畫面看到的大小, 跟真正寫到文件是不一樣的 (因為經過程式的自動縮放), 所以比例正常都要修改的.

留意的是, 圖片的路徑跟正常日常使用的路徑會不一樣, 是使用所謂的"相對路徑" (Relative path), 而起點是論文的主檔案(thesis.tex).

\noindent 例如 (以Windows的路徑為例子):\\
主檔案thesis.tex在: "\verb|C:\thesis\thesis.tex|"\\
圖片A: "\verb|C:\thesis\some_dir\A.png|"\\
圖片B: "\verb|C:\thesis\some_dir\B.png|"\\
使用時以"\verb|./some_dir/A.png|", "\verb|./some_dir/B.png|"的方式來使用, 注意是"$/$"而不是"$\backslash$".

還有檔名的文字中間不要在非格式的位置(如最後的.png, .pdf等)出現`.'這個符號 (如AB.CD-EF.png), 否則有可能會出現潛在的錯誤.

% ------------------------------------------------
\newpage
\StartSubSection{單張}

  \begin{DescriptionFrame}
  \begin{verbatim}
  Path:   圖片位置 (必填)

  Options 設定 (使用','來分隔, 不分先後順序)
    scale:   比例 (選填, 預設: 1.0)
    (1.0: 原大小; 0.x ~ < 1.0: 縮小; > 1.0: 放大)
    (設計上你是可以無限放大, 但還是推薦你使用大圖, 之後縮小)
    caption: 標題 (選填)
    label:   標簽 (選填, 必須要配合caption使用, 否則無效)
    angle:   角度 (選填, 預設: 0度)
    opacity: 背景顏色透明度, 預設使用白色為背景 (選填, 預設: 0.75)
    (0.x ~ < 1.0: 透明; => 1.0: 不透明)

  插入圖片
  \InsertFigure[Options]{Path}

  E.g
    \InsertFigure
    [caption={這 是 標 題}]
      {./figure.png}

    \InsertFigure
      [scale=0.5,
        angle=45,
        caption={這 是 標 題},
        label={this:is:label}]
      {./figure.png}

    每一項資料可以使用斷行來分隔以保持可讀性.
    caption和label必須要使用'{}'才能有空格的句子.

    補充:
        LaTex對SVG檔的支援並不理想, 故推薦先對SVG進行加工,
        如轉成.eps或.pdf (推薦).
  \end{verbatim}
  \end{DescriptionFrame}

  \newpage
  {\bf 效果:}
  \begin{enumerate}
  \item
  {
    只填了圖片位置
    \begin{verbatim}
    \InsertFigure
      {./figure.png}
    \end{verbatim}
    \InsertFigure
      {./example/how-to/write/figure/pic/Cc-by_new.png}
  } % End of \item{}

  \item
  {
    放大比例
    \begin{verbatim}
    \InsertFigure
      [scale=1.5]
      {./figure.png}
    \end{verbatim}
    \InsertFigure
      [scale=1.5]
      {./example/how-to/write/figure/pic/Cc-by_new.png}
  } % End of \item{}

  \item
  {
    縮小比例
    \begin{verbatim}
    \InsertFigure
      [scale=0.5]
      {./figure.png}
    \end{verbatim}
    \InsertFigure
      [scale=0.5]
      {./example/how-to/write/figure/pic/Cc-by_new.png}
  } % End of \item{}

  \newpage
  \item
  {
    增加標題並去掉比例的數字
    \begin{verbatim}
    \InsertFigure
      [caption={Little man}]
      {./figure.png}
    \end{verbatim}
    \InsertFigure
      [caption={Little man}]
      {./example/how-to/write/figure/pic/Cc-by_new.png}
  } % End of \item{}

  \item
  {
    增加標簽
    \begin{verbatim}
    \InsertFigure
      [caption={Little man No.1},
        label={fig:little-man-no.1}]
      {./figure.png}
    \end{verbatim}

    之後可以使用\verb| \RefTo |去引用 \verb| \RefTo{fig:little-man-no.1} |
    \InsertFigure
      [caption={Little man No.1},
        label={fig:little-man-no.1}]
      {./example/how-to/write/figure/pic/Cc-by_new.png}

    e.g: 文中所指的人物一號 (Fig \RefTo{fig:little-man-no.1}).
  } % End of \item{}

  \newpage
  \item
  {
    使用角度去轉45度
    \begin{verbatim}
    \InsertFigure
      [angle=45,
        caption={Little man No.2},
        label={fig:little-man-no.2}]
      {./figure.png}
    \end{verbatim}

    使用\verb| \RefTo |去引用 \verb| \RefTo{fig:little-man-no.2} |
    \InsertFigure
      [angle=45,
        caption={Little man No.2},
        label={fig:little-man-no.2}]
      {./example/how-to/write/figure/pic/Cc-by_new.png}

    e.g: 文中所指的人物二號 (Fig \RefTo{fig:little-man-no.2}).
  } % End of \item{}

  \newpage
  \item
  {
    使用透明度以能看到頁面中的學校浮水印, 不過除非你是使用很小的圖, 否則還是會被你的圖蓋到而會感覺不出來的.\\

    \vspace{2.0cm}

    \begin{verbatim}
    \InsertFigure
      [caption={opacity使用預設}]
      {./figure.png}
    \end{verbatim}

    \InsertFigure
      [scale=0.5,
        caption={opacity使用預設}]
      {./example/abstract/pic/extended-abstract-2.jpg}

    \newpage
    \EmptyLine
    \vspace{2.5cm}

    \begin{verbatim}
    \InsertFigure
      [opacity=0.4,
      caption={opacity使用0.4}]
      {./figure.png}
    \end{verbatim}

    \InsertFigure
      [scale=0.5,
        caption={opacity使用0.4},
        opacity=0.4]
      {./example/abstract/pic/extended-abstract-2.jpg}

    \newpage
    \EmptyLine
    \vspace{7.0cm}

    \InsertFigures
    [caption={opacity使用預設}] %
    {
      {./example/how-to/write/figure/pic/CC-BY-NC.png}
    }%
    {
      {./example/how-to/write/figure/pic/CC-BY-NC-ND.png}
    }

    \vspace{1.0cm}

    \InsertFigures
    [caption={opacity使用0.4},
    opacity=0.4]
    {
      {./example/how-to/write/figure/pic/CC-BY-NC.png}
    }%
    {
      {./example/how-to/write/figure/pic/CC-BY-NC-ND.png}
    }

  } % End of \item{}

  \newpage
  \item
  {
    把圖放在表格中, 這時候是使用\verb|\InsertFigure{}|, 是不能使用caption和label (正常應該用不到這種+寫法, 例如出現`Fig X.X'這種字在table中). (有關表格table的使用, 請參考Chap \RefTo{chapter:how-to:write:table}). 如真的想使用, 則考慮這邊的寫法.

    \begin{verbatim}
    \begin{table}[H]
    \centering
    \begin{tabular}{|c|c|}
      \hline
      \textbf{\underline{Website}} &
        \textbf{\underline{URL}} \\ \hline

      \begin{tabular}[c]{@{}c@{}}
      \includegraphics[scale=0.1]
        {./apple.png} \\ Apple
      \end{tabular} & \url{www.apple.com}  \\ \hline

      \begin{tabular}[c]{@{}c@{}}
      \includegraphics[scale=0.1]
        {./google.png} \\ Google
      \end{tabular} & \url{www.google.com} \\ \hline
    \end{tabular}
    \end{table}
    \end{verbatim}

    \begin{table}[H]
    \centering
    \label{table:how-to:write:figure:insert-figure-into-table}
    \begin{tabular}{|c|c|}
      \hline
      \textbf{\underline{Website}} &
        \textbf{\underline{URL}} \\ \hline

      \begin{tabular}[c]{@{}c@{}}
      \includegraphics[scale=0.1]
       {./example/how-to/write/figure/pic/apple.jpg} \\ Apple
      \end{tabular} & \url{www.apple.com}  \\ \hline

      \begin{tabular}[c]{@{}c@{}}
      \includegraphics[scale=0.1]
        {./example/how-to/write/figure/pic/google.png} \\ Google
      \end{tabular} & \url{www.google.com} \\ \hline
    \end{tabular}
    \end{table}

  } % End of \item{}
  \end{enumerate}

% ------------------------------------------------
\newpage
\StartSubSection{多張}

  如果要同時顯示多張的話, 因為要能一頁版面的範圍內, 同時又要能清楚顯示到你圖中的內容和文字, 大約4張都已經算多的了. 所以真的數量比較多的話, 推薦分別放同不到頁面會比較好閱讀.

  多張是使用\verb|\InsertFigures| ({\bf 注意:} Figure是複數, 有一個`s'), 設計上可插入1$\sim$8張的圖片, 而且寫法會跟插入單張相近.\\

  \begin{DescriptionFrame}
  \begin{verbatim}
  Options 主圖的設定 (使用','來分隔, 不分先後順序)
    perrow:  每一列多少張圖片 (選填, 預設: 1. 最小: 1, 最大: 4)
    caption: 標題 (選填)
    label:   標簽 (選填, 必須要配合caption使用, 否則無效)
    opacity: 背景顏色透明度, 預設使用白色為背景 (選填, 預設: 0.75)
    (0.x ~ < 1.0: 透明; => 1.0: 不透明)

  Figure 1~8: 各張圖片的設定
    設定方式跟使用\InsertFigure是一樣的
    [Figure options] -> [Options]
    {Figure path}  -> {Path}

  插入多張圖片
    \InsertFigures[Options] %
    {
      [Figure options]{Figure path}
    }%
    {
      ...
    }%
    {
      [Figure options]{Figure path}
    }
  ('%'是必須存在的, 以防止被LaTex認為這是新段落)
  \end{verbatim}
  \end{DescriptionFrame}

  \newpage

  {\bf 效果:}
  \begin{enumerate}
  \item
  {
    插入2張圖片, 以1張圖為一列
    \begin{verbatim}
    \InsertFigures
    [caption = {2 figures and 1 figure per row}] %
    {
      {./figure.png}
    }%
    {
      {./figure.png}
    }
    \end{verbatim}
    \InsertFigures
    [caption = {2 figures and 1 figure per row}] %
    {
      {./example/how-to/write/figure/pic/CC-BY-NC.png}
    }%
    {
      {./example/how-to/write/figure/pic/CC-BY-NC-ND.png}
    }
  } % End of \item{}
  \item
  {
    插入2張圖片, 以2張圖為一列
    \begin{verbatim}
    \InsertFigures
    [perrow = 2,
      caption = {2 figures and 2 figures per row}] %
    {
      {./figure.png}
    }%
    {
      {./figure.png}
    }
    \end{verbatim}
    \InsertFigures
    [perrow = 2,
      caption = {2 figures and 2 figures per row}] %
    {
    [caption = {2 figures and 2 figures per row}]%
      {./example/how-to/write/figure/pic/CC-BY-NC.png}
    }%
    {
      {./example/how-to/write/figure/pic/CC-BY-NC-ND.png}
    }
  } % End of \item{}

  \newpage
  \item
  {
    插入3張圖片, 2張圖一列, 並有1張圖轉變角度, 同時主圖跟2張子圖片做了標簽
    \begin{verbatim}
    \InsertFigures
    [perrow = 2,
      caption = {3 figures and 2 figures per row},
      label = {fig:example:mi2:fig1}] %
    {
      [caption = {Figure 1},
      label = {fig:example:mi2:fig1}]
      {./figure.png}
    }%
    {
      [caption = {Figure 2},
      label = {fig:example:mi2:fig2}, angle = -20]
      {./figure.png}
    }%
    {
      [caption = {Figure 3}]
      {./figure.png}
    }
    \end{verbatim}

%    \newpage
    效果會是這樣: \\
    \InsertFigures
    [perrow = 2,%
      caption = {3 figures and 2 figures per row},
      label = {fig:example:mi2:mfig}] %
    {
      [caption = {Figure 1},
      label = {fig:example:mi2:fig1}]
      {./example/how-to/write/figure/pic/CC-BY-NC.png}
    }%
    {
      [caption = {Figure 2},
      label = {fig:example:mi2:fig2},
      angle = -20]
      {./example/how-to/write/figure/pic/CC-BY-NC-ND.png}
    }%
    {
      [caption = {Figure 3}]
      {./example/how-to/write/figure/pic/CC-BY-NC-SA.png}
    }%

    e.g:
    引用主圖 (Fig \RefTo{fig:example:mi2:mfig}) ,
    引用子圖片 (Fig \RefTo{fig:example:mi2:fig1}, Fig \RefTo{fig:example:mi2:fig2}).
  } % End of \item{}

  \newpage
  \item
  {
    插入4張圖片, 2張圖一列, 只有主圖做了標簽.\\
    如果需要不填內容, 但需要圖片的編號的話, 就在caption填寫`{ }'(有空格在中間), 而`{}'則會被認為沒有填寫.
    \begin{verbatim}
    \InsertFigures
    [perrow = 2,
      caption = {4 figures and 2 figures per row},
      label = {fig:example:mi3:mfig}] %
    {
      [caption = { }, label = {fig:example:mi3:fig1}]
      {./figure.png}
    }%
    {
      [caption = {}, label = {fig:example:mi3:fig2}]
      {./figure.png}
    }%
    {
      [caption = { }, label = {fig:example:mi3:fig3}]
      {./figure.png}
    }%
    {
      [caption = { }, label = {fig:example:mi3:fig4}]
      {./figure.png}
    }
    \end{verbatim}

    \InsertFigures
    [perrow = 2,
      caption = {4 figures and 2 figures per row},
      label = {fig:example:mi3:mfig}] %
    {
      [caption = { },
      label = {fig:example:mi3:fig1}]
      {./example/how-to/write/figure/pic/CC-BY-NC.png}
    }%
    {
      [caption = {},
      label = {fig:example:mi3:fig2}]
      {./example/how-to/write/figure/pic/CC-BY-NC-ND.png}
    }%
    {
      [caption = { },
      label = {fig:example:mi3:fig3}]
      {./example/how-to/write/figure/pic/CC-BY-NC-SA.png}
    }%
    {
      [caption = { },
      label = {fig:example:mi3:fig4}]
      {./example/how-to/write/figure/pic/CC-BY-ND.png}
    }

    可以看得出圖片的編號不一樣了\\
    引用主圖 (Fig \RefTo{fig:example:mi3:mfig})\\
    引用子圖片(a) (Fig \RefTo{fig:example:mi3:fig1})\\
    引用子圖片(b) (由於這張圖的caption是沒設定, 所以label無效)\\
    引用子圖片(c) (Fig \RefTo{fig:example:mi3:fig3})\\
    引用子圖片(d) (Fig \RefTo{fig:example:mi3:fig4})
  } % End of \item{}

  %
  \newpage
  \item
  {
    插入8張圖片, 2張圖一列, 只有主圖填了標題.\\
    \begin{verbatim}
    \InsertFigures
    [perrow = 2,
      caption = {8 figures and 2 figures per row}] %
    {[caption = { }]{./figure.png}}%
    {[caption = { }]{./figure.png}}%
    {[caption = { }]{./figure.png}}%
    {[caption = { }]{./figure.png}}%
    {[caption = { }]{./figure.png}}%
    {[caption = { }]{./figure.png}}%
    {[caption = { }]{./figure.png}}%
    {[caption = { }]{./figure.png}}
    \end{verbatim}

    \InsertFigures
    [perrow = 2,
      caption = {8 figures and 2 figures per row}] %
    {[caption = { }]{./example/how-to/write/figure/pic/CC-BY.png}}%
    {[caption = { }]{./example/how-to/write/figure/pic/CC-BY-NC.png}}%
    {[caption = { }]{./example/how-to/write/figure/pic/CC-BY-ND.png}}%
    {[caption = { }]{./example/how-to/write/figure/pic/CC-BY-SA.png}}%
    {[caption = { }]{./example/how-to/write/figure/pic/CC-BY.png}}%
    {[caption = { }]{./example/how-to/write/figure/pic/CC-BY-NC.png}}%
    {[caption = { }]{./example/how-to/write/figure/pic/CC-BY-ND.png}}%
    {[caption = { }]{./example/how-to/write/figure/pic/CC-BY-SA.png}}
  } % End of \item{}

  %
  \newpage
  \item
  {
    插入8張圖片, 3張圖一列, 只有主圖填了標題.\\
    \begin{verbatim}
    \InsertFigures
    [perrow = 3,
      caption = {8 figures and 3 figures per row}] %
    {[caption = { }]{./figure.png}}%
    {[caption = { }]{./figure.png}}%
    {[caption = { }]{./figure.png}}%
    {[caption = { }]{./figure.png}}%
    {[caption = { }]{./figure.png}}%
    {[caption = { }]{./figure.png}}%
    {[caption = { }]{./figure.png}}%
    {[caption = { }]{./figure.png}}
    \end{verbatim}

    \InsertFigures
    [perrow = 3,
      caption = {8 figures and 3 figures per row}] %
    {[caption = { }]{./example/how-to/write/figure/pic/CC-BY.png}}%
    {[caption = { }]{./example/how-to/write/figure/pic/CC-BY.png}}%
    {[caption = { }]{./example/how-to/write/figure/pic/CC-BY.png}}%
    {[caption = { }]{./example/how-to/write/figure/pic/CC-BY.png}}%
    {[caption = { }]{./example/how-to/write/figure/pic/CC-BY.png}}%
    {[caption = { }]{./example/how-to/write/figure/pic/CC-BY.png}}%
    {[caption = { }]{./example/how-to/write/figure/pic/CC-BY.png}}%
    {[caption = { }]{./example/how-to/write/figure/pic/CC-BY.png}}
  } % End of \item{}

  %
  \newpage
  \item
  {
    插入8張圖片, 4張圖一列, 只有主圖填了標題.\\
    \begin{verbatim}
    \InsertFigures
    [perrow = 4,
      caption = {8 figures and 4 figures per row}] %
    {[caption = { }]{./figure.png}}%
    {[caption = { }]{./figure.png}}%
    {[caption = { }]{./figure.png}}%
    {[caption = { }]{./figure.png}}%
    {[caption = { }]{./figure.png}}%
    {[caption = { }]{./figure.png}}%
    {[caption = { }]{./figure.png}}%
    {[caption = { }]{./figure.png}}
    \end{verbatim}

    \InsertFigures
    [perrow = 4,
      caption = {8 figures and 4 figures per row}] %
    {[caption = { }]{./example/how-to/write/figure/pic/CC-BY.png}}%
    {[caption = { }]{./example/how-to/write/figure/pic/CC-BY-NC.png}}%
    {[caption = { }]{./example/how-to/write/figure/pic/CC-BY-ND.png}}%
    {[caption = { }]{./example/how-to/write/figure/pic/CC-BY-SA.png}}%
    {[caption = { }]{./example/how-to/write/figure/pic/CC-BY.png}}%
    {[caption = { }]{./example/how-to/write/figure/pic/CC-BY-NC.png}}%
    {[caption = { }]{./example/how-to/write/figure/pic/CC-BY-ND.png}}%
    {[caption = { }]{./example/how-to/write/figure/pic/CC-BY-SA.png}}
  } % End of \item{}
  \end{enumerate}
% ------------------------------------------------
\EndChapter
% ------------------------------------------------
