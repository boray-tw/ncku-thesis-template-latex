% ------------------------------------------------
\StartChapter{LaTex編寫教學}
% ------------------------------------------------

% ------------------------------------------------
\StartSection{基本介紹 Introduction}{chapter:how-to:write:intro}
% ------------------------------------------------

這教學包含了原LaTex和本模版特有的語法的使用方式和例子. (真正完完整整的LaTex教學手冊可不只單單幾百頁的厚度, 所以減少大家的時間, 所以本模版教學只講一些幾乎大家100\%會需要使用的語法).

請注意原LaTex語法會以英文小寫來顯示(\verb|\aabbcc|); 而本模版特有的語法會以英文大小寫混合(\verb|\AaBbCc|, 第一個字必定以大寫來顯示), 由於這些特有語法\textbf{不是}原LaTex的語法, 所以不能直接應用在非本模版的LaTex檔案上.

抄襲就是學習的第一步 (如同我們小時候去抄襲父母走路一樣), 所以本模版有留下了一些範本 (在`./context'下)以方便大家開始第一步, 之後就要靠大家自己的努力和實作, 再加上自己的探索能力了.

%\newpage
有問題的話, 可以有以下的地方找尋答案 (請使用這順序):
\begin{enumerate}
  \item 請一步一步增加內容, 如發生錯誤, 就把剛剛新增的內容拿掉, 以找出錯誤的地方
  \item 直接研究在模版的LaTex寫法 (在 './example' 以下的所有檔案)
  \item 查問懂得LaTex的老師和同學
  \item 去LaTex的Wikibook \RefBib{web:latex:wikibooks}\\
        這邊有大量的例子, 但是這些例子都是獨立的, 所以潛在語法混合後的會發生沖突的可能性; 另外都十分推薦去讀 '大家來學LaTeX' \RefBib{web:latex:latex123}
  \item 請求Google老師
\end{enumerate}

另外, 如果覺得本教學還缺少了什麼說明, 請告知.

% ------------------------------------------------

% Section
\newpage% ------------------------------------------------
\StartSection{基本語法 Basic syntax}{chapter:how-to:write:basic}
% ------------------------------------------------

這邊會講解一些最基本的功能.

% ------------------------------------------------
% ------------------------------------------------
\StartSubSection{字體變化}

\begin{itemize}
  \item
  {
    正常

    這是文字 This is text
  } % End of \item{}

  \item
  {
    粗體

    寫法:
    \begin{DescriptionFrame}
    \verb|\textbf{這是文字 This is text}|
    \end{DescriptionFrame}

    效果: \textbf{這是文字 This is text}
  } % End of \item{}

  \item
  {
    斜体

    寫法:
    \begin{DescriptionFrame}
    \verb|\textit{這是文字 This is text}|
    \end{DescriptionFrame}

    效果: \textit{這是文字 This is text}\\
    (中文的斜体可能會並不太明顯)
  } % End of \item{}
\end{itemize}
% ------------------------------------------------


% ------------------------------------------------
\newpage% ------------------------------------------------
\StartSubSection{清單 List Structures}

  日常的清單主要有3種:

\begin{itemize}
  \item
  {
    數字

    可以有2種寫法, 使用\verb|\item xxxx|來只寫一行, 或是用\verb|{...}|可把內容包起來.\\

    \begin{DescriptionFrame}
    \begin{verbatim}
      \begin{enumerate}
      \item Item1

      \item Item2

      \item
      {
        Item3

        Item3's context
      }

      \item
      {
        Item4

        Item4's context
      }
      \end{enumerate}
    \end{verbatim}
    \end{DescriptionFrame}

    效果:
    \begin{enumerate}
      \item Item1

      \item Item2

      \item
      {
        Item3

        Item3's context
      }

      \item
      {
        Item4

        Item4's context
      }
    \end{enumerate}
  } % End of \item{}

  \newpage
  \item
  {
    符號

    \begin{DescriptionFrame}
    \begin{verbatim}
      \begin{itemize}
      \item Item1

      \item Item2

      \item
      {
        Item3

        Item3's context
      }

      \item
      {
        Item4

        Item4's context
      }
      \end{itemize}
    \end{verbatim}
    \end{DescriptionFrame}

    效果:
    \begin{itemize}
      \item Item1

      \item Item2

      \item
      {
        Item3

        Item3's context
      }

      \item
      {
        Item4

        Item4's context
      }
    \end{itemize}
  } % End of \item{}

  \newpage
  \item
  {
    文字

    可以有2種寫法, 使用\verb|\item[xxxx] xxxx|來只寫一行,\\
    或是用\verb|\hfill \\|把內容放到第2行才開始.\\

    \begin{DescriptionFrame}
    \begin{verbatim}
      \begin{description}
      \item[Item1] Item1's context
      \item[Item2] Item2's context
      \item[Item3] \hfill \\
        Item3's context
      \end{description}
    \end{verbatim}
    \end{DescriptionFrame}

    效果:
    \begin{description}
      \item[Item1] Item1's context
      \item[Item2] Item2's context
      \item[Item3] \hfill \\
      Item3's context
    \end{description}
  } % End of \item{}

  \newpage
  \item
  {
    巢狀表單

    表單應該最多只會用到第4層, 但是其實當你需要用到第3層時, 這時候你應該考慮的不是怎使用表單, 而是要怎換另外一種寫法了.\\

    \begin{DescriptionFrame}
    \begin{verbatim}
      \begin{enumerate}
        \item
        {
          Level-1 Item 1
          \begin{enumerate}
            \item Nested Item 1

            \item
            {
              Level-2 Item 2

              \begin{enumerate}
              \item
              {
                Level-3 Item 1
                \begin{enumerate}
                  \item Level-4 Item 1
                  \item Level-4 Item 2
                \end{enumerate}
              }
              \item Level-3 Item 2
              \end{enumerate}
            }
          \end{enumerate}
        }
      \end{enumerate}

      \begin{itemize}
        \item
        {
          Level-1 Item 1

          \begin{itemize}
            \item
            {
              Level-2 Item 2
              \begin{itemize}
                \item Level-3 Item 1
                \item Level-3 Item 2
              \end{itemize}
            }
            \item Level-2 Item 2
          \end{itemize}
        }
      \end{itemize}
    \end{verbatim}
    \end{DescriptionFrame}

    效果:
    \begin{enumerate}
      \item
      {
        Level-1 Item 1
        \begin{enumerate}
          \item Nested Item 1

          \item
          {
            Level-2 Item 2

            \begin{enumerate}
              \item
              {
                Level-3 Item 1

                \begin{enumerate}
                  \item Level-4 Item 1
                  \item Level-4 Item 2
                \end{enumerate}
              }

              \item Level-3 Item 2
            \end{enumerate}
          }
        \end{enumerate}
      }
    \end{enumerate}

    \begin{itemize}
      \item
      {
        Level-1 Item 1

        \begin{itemize}
        \item
        {
          Level-2 Item 2

          \begin{itemize}
          \item Level-3 Item 1
          \item Level-3 Item 2
          \end{itemize}
        }

        \item Level-2 Item 2
        \end{itemize}
      }
    \end{itemize}
  } % End of \item{}
\end{itemize}
% ------------------------------------------------


% ------------------------------------------------
\newpage% ------------------------------------------------
\StartSubSection{標記 Label}{chapter:how-to:write:label}
標記(Label)是指給某項東西(如圖, 表格, 段落, chapter等)一個用來記憶的名字, 主要用來在引用時可以用來指定它. 使用方式是:\\

  \begin{DescriptionFrame}
  \begin{verbatim}
    \label{ ... some text here for your label ...} % 設定Label

    e.g
    \label{fig:introduction:fig1} % 設定Label
    \RefTo{fig:introduction:fig1} % 引用Label
  \end{verbatim}
  \end{DescriptionFrame}

\noindent Label的名字是可以任何輸入的文字, 但是為了方便記憶, 會固定以一個名字起頭, 再以段落/章節的方式來分隔.\\

\noindent 在例子中`fig:introduction:fig1':\\
以`fig'起頭: 即是目標是一張圖像(figure).\\
以`introduction'為章節: 即是目標放在introduction這一章中.\\
最後`fig1': 這張圖像的名字為`fig1'.\\

\noindent 同樣其他方便記憶的目標起頭例如: `website', `table', `chapter', `section', `paper', etc.\\

\noindent 本模板提供的一些功能內, 已經把這功能包含進來了.

\newpage
\StartSubSection{引用 Reference}
因為原本LaTex的引用語法可以引用很多東西, 所以可能會混亂不知道自己在引用什麼, 故本模板提供幾個語法來取代那些語法. (但是如果你是懂得原LaTex的寫法(\verb|\ref{}, \cite{}, etc.|), 都可以直接使用原本的寫法, 其實是同一個東西.)\\

  \begin{DescriptionFrame}
  \begin{verbatim}
    引用 公式(Equation)
    \RefEquation{...}   直接顯示章節和它的號碼, 如: X.X
    \RefEquationB{...}  顯示時多了'()', 如: (X.X)

    引用 參考資料(References)
    \RefBib{...}   顯示號碼, 會加上'[]', 如: [X]

    引用 頁碼
    \RefPage{...}  顯示目標的頁碼, 如: X

    引用 其他任何的東西: 如圖片, 表格,
          chapter, section, subsection, etc.
    \RefTo{...}
      顯示章節和它的號碼, 如: X.X
      所以要手動在引用部份加上 fig, table, chap等一些字眼
  \end{verbatim}
  \end{DescriptionFrame}

由於label寫在LaTex中, 而產生出來的後的文件是看不到的, 所以沒法簡單講解來說明, 所以可以參考後面的一些章節, 其內容會有一些例子會方便理解.

例子:
\begin{itemize}
  \item 圖片 - 可參考P. \RefPage{fig:example:mi2:mfig}.

  \item 表格 - 可參考P. \RefPage{chapter:how-to:write:table:label-example}.

  \item 公式(Equation) - 可參考P. \RefPage{chapter:how-to:write:equation:label-example}.
\end{itemize}

% ------------------------------------------------


% ------------------------------------------------
\newpage% ------------------------------------------------
\StartSubSection{註解 Comment}{chapter:how-to:write:comment}
% ------------------------------------------------

編寫任何內容時, 都會有一些作輔助用的內容, 這些內容正常不一定是用來顯示給別人看, 而是給自己作一些記憶用的.\\

但是在Word中所寫的任何內容, 正常都是寫來公開的, 而一些個人後備輔助用的資料就會寫在另一個檔案中; 但在LaTex中可以一同把這些資料寫在同一個檔案中, 但可指定不顯示, 這些叫註解(Comment).\\

\begin{DescriptionFrame}
  \begin{verbatim}
    單行註解 (在第一個字使用'%'即可)

    % 註解內容 1
    % 註解內容 2
    顯示內容 1
       ...
    顯示內容 2
       ...
    多行註解 (把一個範圍內的內容為註解)

    \begin{comment}
    % 註解內容 1
    % 註解內容 2
    \end{comment}
    顯示內容 1
       ...
    顯示內容 2
       ...
  \end{verbatim}
\end{DescriptionFrame}


% ------------------------------------------------
\newpage% ------------------------------------------------
\StartSubSection{引用別的LaTex檔}

正常在編寫Word時, 都會把所有內容寫在同一個.doc中 (當然你都可能原本就喜好分開檔案來寫), 但在LaTex中這行為就不常見, 當內容很巨量的時候就更不用講, 這本模版更是其一例子.\\

  \begin{DescriptionFrame}
  \begin{verbatim}
    引用的方式
    \input{ ... 檔案位置 ... }

    如現在你的檔案為:
    thesis.tex (主檔案)
    a.tex
    b.tex

    那要引用a.tex和b.tex時
    在thesis.tex中要寫
    \input{./a.tex}
    \input{./b.tex}
  \end{verbatim}
  \end{DescriptionFrame}

如果還是不明白的話, 可以參考`./example'中的引用方式.

% ------------------------------------------------


\newpage% ------------------------------------------------
\StartSection{章節 Chapter/Section}{chapter:how-to:write:chapter-section}
% ------------------------------------------------

編寫任何的文章, 都會使用不同的章節來把內容進行分區. 例如這模板預設的樣子為:
\begin{DescriptionFrame}
  \vspace{0.2cm}
  \centerline{\LARGE Chapter X}
  \vspace{0.3cm}
  \centerline{\LARGE 這是標題}

  \vspace{0.5cm}
  \mbox{\Large X.1 節標題}\\
  \mbox{\hspace{1.2cm}內容 ...}

  \vspace{0.3cm}
  \mbox{\large X.1.1 小節標題}\\
  \mbox{\hspace{1.2cm}內容 ...}

  \vspace{0.3cm}
  \mbox{\large 小小節標題}\\
  \mbox{\hspace{1.2cm}內容 ...}
\end{DescriptionFrame}

所以針對這些功能, 本模板提供:
\begin{DescriptionFrame}
  \begin{verbatim}
    主要章節
    Title: 標題 (必填)
    Label: 標簽 (選填)
    \StartChapter{ Title }{ Label }
    \EndChapter % 用來保證你的內容在這Chapter內

    節
    Title: 標題 (必填)
    Label: 標簽 (選填)
    \StartSection{ Title }{ Label }

    小節
    Title: 標題 (必填)
    Label: 標簽 (選填)
    \StartSubSection{ Title }{ Label }

    小小節
    Title: 標題 (必填)
    Label: 標簽 (選填)
    \StartSubSubSection{ Title }{ Label }
  \end{verbatim}
\end{DescriptionFrame}

所以針對剛剛的例子, 它的LaTex寫法為:\\

\begin{DescriptionFrame}
  \begin{verbatim}
    \StartChapter{這是標題}

    \StartSection{節標題}
    內容 ...

    \StartSubSection{小節標題}
    內容 ...

    \StartSubSubSection{小小節標題}
    內容 ...

    \EndChapter
  \end{verbatim}
\end{DescriptionFrame}

\newpage% ------------------------------------------------
\StartSection{圖片 Figure}{chapter:how-to:write:figure}
% ------------------------------------------------

\StartSubSection{圖片小知識}

圖片為兩類別: 點陣圖或向量圖 (參考 Fig \RefTo{fig:how-to:write:figure:format}) .\\

  \InsertFigure
    [scale=0.3,
      caption={圖片類別},
      label={fig:how-to:write:figure:format}]
    {./example/how-to/write/figure/pic/graph-format.png}

點陣圖(如.jpg, .gif,.png, .tiff)在相機和網頁中十分常見, 優點是幾乎能應用在不同地方/工具, 缺點在放大縮小時會出現失真的情況, 所以為了更清晰, 則要更高的解析度的圖片, 這時候就會大大增加圖片大小, 而這大小會直接影響所產生出來論文PDF的大小.

向量圖(如.pdf, .eps, .svg)在學術界內用來放在論文中是非常常見. 優點是不會因放大縮小而造成內容變型, 所以是十分有用的. 缺點是必須使用一些特定的工具才能顯示或產生出來. 而LaTex主要使用這一類圖檔. 但LaTex對SVG的支援十分不好, 故這模版沒無提供插入SVG檔, 故強烈推薦使用PDF為主要的格式.

PDF可同時為點陣圖或向量圖, 主要是看你提供什麼圖片格式來轉成PDF. 另一種Windows增加型中繼檔(.emf)都同樣是點陣圖或向量圖, 而由於Word是沒法直接插入PDF/EPS/SVG的圖檔, 所以需先轉成這格式, 以保留向量圖的品質.

% ------------------------------------------------
\newpage
\StartSubSection{轉換格式}

以下為一些已知的方式可把圖像轉成向量圖格式, 按鈕的位置有可能不一樣, 但應該都會在那些地方中. (以下使用工具版本為: Adobe Acrobat XI Pro, Adobe Illustor CS5, Adobe Photoshop CS5, Microsoft Visio Professional 2013, Microsoft Office Professional 2013 [Excel/PowerPoint])

\begin{description}
  \item[SVG - > PDF] \hfill\\
    如果你有安裝學校的Adobe Acrobat Pro (沒有的話都推薦你安裝. 因為交給圖書館的電子檔時, 你起碼要對PDF檔上鎖), 直接對那個SVG檔右鍵, 就會有`轉換成Adobe PDF' 這個選項.

  \item[Adobe Illustor - > EMF] \hfill\\
    `檔案' -> `轉存' -> 格式拉到最下面就有了.

  \item[Adobe Illustor - > PDF] \hfill\\
    `檔案' -> `另存新檔' -> 在格式中間位置 -> 如果你的是成品的話, 則可考慮把 `保留Illustrator編輯能力', `內嵌頁面縮圖'都拿掉以減少PDF檔的大小.

  \item[Adobe Illustor - > SVG] \hfill\\
    `檔案' -> `另存新檔' -> 在格式最底的位置 -> 如果你想不到有什麼設定, 直接按`確定'就好了. 同時都推薦在`影像'的選項設定為`嵌入'以去掉任何影像有位置不對而造成SVG檔有什麼問題.

  \item[Adobe Photoshop - > PDF] \hfill\\
    `檔案' -> `另存新檔' -> 在格式中下的位置 -> 如果你的是成品的話, 則可考慮把 `保留Photoshop編輯能力'拿掉以減少PDF檔的大小.

  \item[Visio - > PDF] \hfill\\
    `檔案' -> `匯出' -> `建立PDF/XPS'.

  \item[Visio - > SVG] \hfill\\
    `檔案' -> `匯出' -> `變更檔案類型' -> `SVG可縮放向量圖形' -> 下面的`另存新檔'.

  \item[Visio - > EMF] \hfill\\
    `檔案' -> `匯出' -> `變更檔案類型' -> `EMF 增加型中繼檔' -> 下面的`另存新檔'.

  \newpage
  \item[PowerPoint - > EMF] \hfill
    \begin{enumerate}
      \item `檔案' -> `匯出' -> `變更檔案類型' -> `儲存成其他檔案類型'
      \item 下面的`另存新檔' -> 在格式最底部位置選擇`Windows增加型中繼檔' -> `僅此投影片'即可.
      \item 當然都可選擇`所有投影片', 只是`僅此投影片'即可儲存跟PowerPoint同檔名的EMF檔. 而`所有投影片'會把多張的EMF檔放在一個資料夾中.
    \end{enumerate}

  \item[Excel - > PDF] Excel可以把所做出來的圖表轉成PDF內容.
    \begin{enumerate}
      \item 在Excel中對某個圖表按一下左鍵, 之後 `檔案' -> `匯出' -> `建立PDF/XPS'.
      \item 所做出來的PDF應該是一張A4, 所以要做裁切. 打開那個PDF, 右上方有一個`工具' -> 右邊多了一個工具列.
      \item 按`裁切', 之後在圖中的任何地方按2下左鍵, 把`移除白色邊距'打勾 (如右圖中沒任何反應, 重新打勾一下看看), 之後按`確定'即可.
      \item 按`Ctrl + S'來儲存即可.
    \end{enumerate}

  \item[Excel - > EMF] Excel沒有任何直接方式把圖表轉成EMF, 但有方法來間接轉換.
    \begin{enumerate}
      \item 先做一次\textbf{Excel - > PDF}的方法.
      \item `檔案' -> `另存新檔' -> 在格式中間位置選擇`PowerPoint簡報(*.pptx)'.
      \item 之後做一次\textbf{PowerPoint - > EMF}的方法即可.
    \end{enumerate}
\end{description}

\noindent \textbf{注意:} 一些系所可能會使用自己的程式或一些工具以製造出SVG檔, 但以我經驗發現有時候有些工具所產出的SVG檔並不能在日常的工具或瀏覽器正常顯示 (應該是SVG中的XML有某程度的內容跟公認的內容不太一樣所造成的). 所以我會推薦先檢查這SVG檔是否在手上的工具或程式都顯示正常, 之後再把這SVG轉成PDF, 以方便匯入到論文之中, 同時能保證這SVG中的內容沒有出現任何變型或錯誤.

% ------------------------------------------------
\newpage
\StartSubSection{使用介紹}

插入圖片其實有很多的玩法, 但是在畢業論文中, 它的放置位置則是非常固定的, 都是以中間為主, 並插入單/多張圖片. 因為圖片位置都是固定的, 所以本模版針對了插入單張或多張來設計, 之後的章節會對這2個方式的使用作詳細說明.

要注意的是, 圖片在畫面看到的大小, 跟真正寫到文件是不一樣的 (因為經過程式的自動縮放), 所以比例正常都要修改的.

留意的是, 圖片的路徑跟正常日常使用的路徑會不一樣, 是使用所謂的"相對路徑" (Relative path), 而起點是論文的主檔案(thesis.tex).

\noindent 例如 (以Windows的路徑為例子):\\
主檔案thesis.tex在: "\verb|C:\thesis\thesis.tex|"\\
圖片A: "\verb|C:\thesis\some_dir\A.png|"\\
圖片B: "\verb|C:\thesis\some_dir\B.png|"\\
使用時以"\verb|./some_dir/A.png|", "\verb|./some_dir/B.png|"的方式來使用, 注意是"$/$"而不是"$\backslash$".

還有檔名的文字中間不要在非格式的位置(如最後的.png, .pdf等)出現`.'這個符號 (如AB.CD-EF.png), 否則有可能會出現潛在的錯誤.

% ------------------------------------------------
\newpage
\StartSubSection{單張}

  \begin{DescriptionFrame}
  \begin{verbatim}
  Path:   圖片位置 (必填)

  Options 設定 (使用','來分隔, 不分先後順序)
    scale:   比例 (選填, 預設: 1.0)
    (1.0: 原大小; 0.x ~ < 1.0: 縮小; > 1.0: 放大)
    (設計上你是可以無限放大, 但還是推薦你使用大圖, 之後縮小)
    caption: 標題 (選填)
    label:   標簽 (選填, 必須要配合caption使用, 否則無效)
    angle:   角度 (選填, 預設: 0度)
    opacity: 背景顏色透明度, 預設使用白色為背景 (選填, 預設: 0.75)
    (0.x ~ < 1.0: 透明; => 1.0: 不透明)

  插入圖片
  \InsertFigure[Options]{Path}

  E.g
    \InsertFigure
    [caption={這 是 標 題}]
      {./figure.png}

    \InsertFigure
      [scale=0.5,
        angle=45,
        caption={這 是 標 題},
        label={this:is:label}]
      {./figure.png}

    每一項資料可以使用斷行來分隔以保持可讀性.
    caption和label必須要使用'{}'才能有空格的句子.

    補充:
        LaTex對SVG檔的支援並不理想, 故推薦先對SVG進行加工,
        如轉成.eps或.pdf (推薦).
  \end{verbatim}
  \end{DescriptionFrame}

  \newpage
  {\bf 效果:}
  \begin{enumerate}
  \item
  {
    只填了圖片位置
    \begin{verbatim}
    \InsertFigure
      {./figure.png}
    \end{verbatim}
    \InsertFigure
      {./example/how-to/write/figure/pic/Cc-by_new.png}
  } % End of \item{}

  \item
  {
    放大比例
    \begin{verbatim}
    \InsertFigure
      [scale=1.5]
      {./figure.png}
    \end{verbatim}
    \InsertFigure
      [scale=1.5]
      {./example/how-to/write/figure/pic/Cc-by_new.png}
  } % End of \item{}

  \item
  {
    縮小比例
    \begin{verbatim}
    \InsertFigure
      [scale=0.5]
      {./figure.png}
    \end{verbatim}
    \InsertFigure
      [scale=0.5]
      {./example/how-to/write/figure/pic/Cc-by_new.png}
  } % End of \item{}

  \newpage
  \item
  {
    增加標題並去掉比例的數字
    \begin{verbatim}
    \InsertFigure
      [caption={Little man}]
      {./figure.png}
    \end{verbatim}
    \InsertFigure
      [caption={Little man}]
      {./example/how-to/write/figure/pic/Cc-by_new.png}
  } % End of \item{}

  \item
  {
    增加標簽
    \begin{verbatim}
    \InsertFigure
      [caption={Little man No.1},
        label={fig:little-man-no.1}]
      {./figure.png}
    \end{verbatim}

    之後可以使用\verb| \RefTo |去引用 \verb| \RefTo{fig:little-man-no.1} |
    \InsertFigure
      [caption={Little man No.1},
        label={fig:little-man-no.1}]
      {./example/how-to/write/figure/pic/Cc-by_new.png}

    e.g: 文中所指的人物一號 (Fig \RefTo{fig:little-man-no.1}).
  } % End of \item{}

  \newpage
  \item
  {
    使用角度去轉45度
    \begin{verbatim}
    \InsertFigure
      [angle=45,
        caption={Little man No.2},
        label={fig:little-man-no.2}]
      {./figure.png}
    \end{verbatim}

    使用\verb| \RefTo |去引用 \verb| \RefTo{fig:little-man-no.2} |
    \InsertFigure
      [angle=45,
        caption={Little man No.2},
        label={fig:little-man-no.2}]
      {./example/how-to/write/figure/pic/Cc-by_new.png}

    e.g: 文中所指的人物二號 (Fig \RefTo{fig:little-man-no.2}).
  } % End of \item{}

  \newpage
  \item
  {
    使用透明度以能看到頁面中的學校浮水印, 不過除非你是使用很小的圖, 否則還是會被你的圖蓋到而會感覺不出來的.\\

    \vspace{2.0cm}

    \begin{verbatim}
    \InsertFigure
      [caption={opacity使用預設}]
      {./figure.png}
    \end{verbatim}

    \InsertFigure
      [scale=0.5,
        caption={opacity使用預設}]
      {./example/abstract/pic/extended-abstract-2.jpg}

    \newpage
    \EmptyLine
    \vspace{2.5cm}

    \begin{verbatim}
    \InsertFigure
      [opacity=0.4,
      caption={opacity使用0.4}]
      {./figure.png}
    \end{verbatim}

    \InsertFigure
      [scale=0.5,
        caption={opacity使用0.4},
        opacity=0.4]
      {./example/abstract/pic/extended-abstract-2.jpg}

    \newpage
    \EmptyLine
    \vspace{7.0cm}

    \InsertFigures
    [caption={opacity使用預設}] %
    {
      {./example/how-to/write/figure/pic/CC-BY-NC.png}
    }%
    {
      {./example/how-to/write/figure/pic/CC-BY-NC-ND.png}
    }

    \vspace{1.0cm}

    \InsertFigures
    [caption={opacity使用0.4},
    opacity=0.4]
    {
      {./example/how-to/write/figure/pic/CC-BY-NC.png}
    }%
    {
      {./example/how-to/write/figure/pic/CC-BY-NC-ND.png}
    }

  } % End of \item{}

  \newpage
  \item
  {
    把圖放在表格中, 這時候是使用\verb|\InsertFigure{}|, 是不能使用caption和label (正常應該用不到這種+寫法, 例如出現`Fig X.X'這種字在table中). (有關表格table的使用, 請參考Chap \RefTo{chapter:how-to:write:table}). 如真的想使用, 則考慮這邊的寫法.

    \begin{verbatim}
    \begin{table}[H]
    \centering
    \begin{tabular}{|c|c|}
      \hline
      \textbf{\underline{Website}} &
        \textbf{\underline{URL}} \\ \hline

      \begin{tabular}[c]{@{}c@{}}
      \includegraphics[scale=0.1]
        {./apple.png} \\ Apple
      \end{tabular} & \url{www.apple.com}  \\ \hline

      \begin{tabular}[c]{@{}c@{}}
      \includegraphics[scale=0.1]
        {./google.png} \\ Google
      \end{tabular} & \url{www.google.com} \\ \hline
    \end{tabular}
    \end{table}
    \end{verbatim}

    \begin{table}[H]
    \centering
    \label{table:how-to:write:figure:insert-figure-into-table}
    \begin{tabular}{|c|c|}
      \hline
      \textbf{\underline{Website}} &
        \textbf{\underline{URL}} \\ \hline

      \begin{tabular}[c]{@{}c@{}}
      \includegraphics[scale=0.1]
       {./example/how-to/write/figure/pic/apple.jpg} \\ Apple
      \end{tabular} & \url{www.apple.com}  \\ \hline

      \begin{tabular}[c]{@{}c@{}}
      \includegraphics[scale=0.1]
        {./example/how-to/write/figure/pic/google.png} \\ Google
      \end{tabular} & \url{www.google.com} \\ \hline
    \end{tabular}
    \end{table}

  } % End of \item{}
  \end{enumerate}

% ------------------------------------------------
\newpage
\StartSubSection{多張}

  如果要同時顯示多張的話, 因為要能一頁版面的範圍內, 同時又要能清楚顯示到你圖中的內容和文字, 大約4張都已經算多的了. 所以真的數量比較多的話, 推薦分別放同不到頁面會比較好閱讀.

  多張是使用\verb|\InsertFigures| ({\bf 注意:} Figure是複數, 有一個`s'), 設計上可插入1$\sim$8張的圖片, 而且寫法會跟插入單張相近.\\

  \begin{DescriptionFrame}
  \begin{verbatim}
  Options 主圖的設定 (使用','來分隔, 不分先後順序)
    perrow:  每一列多少張圖片 (選填, 預設: 1. 最小: 1, 最大: 4)
    caption: 標題 (選填)
    label:   標簽 (選填, 必須要配合caption使用, 否則無效)
    opacity: 背景顏色透明度, 預設使用白色為背景 (選填, 預設: 0.75)
    (0.x ~ < 1.0: 透明; => 1.0: 不透明)

  Figure 1~8: 各張圖片的設定
    設定方式跟使用\InsertFigure是一樣的
    [Figure options] -> [Options]
    {Figure path}  -> {Path}

  插入多張圖片
    \InsertFigures[Options] %
    {
      [Figure options]{Figure path}
    }%
    {
      ...
    }%
    {
      [Figure options]{Figure path}
    }
  ('%'是必須存在的, 以防止被LaTex認為這是新段落)
  \end{verbatim}
  \end{DescriptionFrame}

  \newpage

  {\bf 效果:}
  \begin{enumerate}
  \item
  {
    插入2張圖片, 以1張圖為一列
    \begin{verbatim}
    \InsertFigures
    [caption = {2 figures and 1 figure per row}] %
    {
      {./figure.png}
    }%
    {
      {./figure.png}
    }
    \end{verbatim}
    \InsertFigures
    [caption = {2 figures and 1 figure per row}] %
    {
      {./example/how-to/write/figure/pic/CC-BY-NC.png}
    }%
    {
      {./example/how-to/write/figure/pic/CC-BY-NC-ND.png}
    }
  } % End of \item{}
  \item
  {
    插入2張圖片, 以2張圖為一列
    \begin{verbatim}
    \InsertFigures
    [perrow = 2,
      caption = {2 figures and 2 figures per row}] %
    {
      {./figure.png}
    }%
    {
      {./figure.png}
    }
    \end{verbatim}
    \InsertFigures
    [perrow = 2,
      caption = {2 figures and 2 figures per row}] %
    {
    [caption = {2 figures and 2 figures per row}]%
      {./example/how-to/write/figure/pic/CC-BY-NC.png}
    }%
    {
      {./example/how-to/write/figure/pic/CC-BY-NC-ND.png}
    }
  } % End of \item{}

  \newpage
  \item
  {
    插入3張圖片, 2張圖一列, 並有1張圖轉變角度, 同時主圖跟2張子圖片做了標簽
    \begin{verbatim}
    \InsertFigures
    [perrow = 2,
      caption = {3 figures and 2 figures per row},
      label = {fig:example:mi2:fig1}] %
    {
      [caption = {Figure 1},
      label = {fig:example:mi2:fig1}]
      {./figure.png}
    }%
    {
      [caption = {Figure 2},
      label = {fig:example:mi2:fig2}, angle = -20]
      {./figure.png}
    }%
    {
      [caption = {Figure 3}]
      {./figure.png}
    }
    \end{verbatim}

%    \newpage
    效果會是這樣: \\
    \InsertFigures
    [perrow = 2,%
      caption = {3 figures and 2 figures per row},
      label = {fig:example:mi2:mfig}] %
    {
      [caption = {Figure 1},
      label = {fig:example:mi2:fig1}]
      {./example/how-to/write/figure/pic/CC-BY-NC.png}
    }%
    {
      [caption = {Figure 2},
      label = {fig:example:mi2:fig2},
      angle = -20]
      {./example/how-to/write/figure/pic/CC-BY-NC-ND.png}
    }%
    {
      [caption = {Figure 3}]
      {./example/how-to/write/figure/pic/CC-BY-NC-SA.png}
    }%

    e.g:
    引用主圖 (Fig \RefTo{fig:example:mi2:mfig}) ,
    引用子圖片 (Fig \RefTo{fig:example:mi2:fig1}, Fig \RefTo{fig:example:mi2:fig2}).
  } % End of \item{}

  \newpage
  \item
  {
    插入4張圖片, 2張圖一列, 只有主圖做了標簽.\\
    如果需要不填內容, 但需要圖片的編號的話, 就在caption填寫`{ }'(有空格在中間), 而`{}'則會被認為沒有填寫.
    \begin{verbatim}
    \InsertFigures
    [perrow = 2,
      caption = {4 figures and 2 figures per row},
      label = {fig:example:mi3:mfig}] %
    {
      [caption = { }, label = {fig:example:mi3:fig1}]
      {./figure.png}
    }%
    {
      [caption = {}, label = {fig:example:mi3:fig2}]
      {./figure.png}
    }%
    {
      [caption = { }, label = {fig:example:mi3:fig3}]
      {./figure.png}
    }%
    {
      [caption = { }, label = {fig:example:mi3:fig4}]
      {./figure.png}
    }
    \end{verbatim}

    \InsertFigures
    [perrow = 2,
      caption = {4 figures and 2 figures per row},
      label = {fig:example:mi3:mfig}] %
    {
      [caption = { },
      label = {fig:example:mi3:fig1}]
      {./example/how-to/write/figure/pic/CC-BY-NC.png}
    }%
    {
      [caption = {},
      label = {fig:example:mi3:fig2}]
      {./example/how-to/write/figure/pic/CC-BY-NC-ND.png}
    }%
    {
      [caption = { },
      label = {fig:example:mi3:fig3}]
      {./example/how-to/write/figure/pic/CC-BY-NC-SA.png}
    }%
    {
      [caption = { },
      label = {fig:example:mi3:fig4}]
      {./example/how-to/write/figure/pic/CC-BY-ND.png}
    }

    可以看得出圖片的編號不一樣了\\
    引用主圖 (Fig \RefTo{fig:example:mi3:mfig})\\
    引用子圖片(a) (Fig \RefTo{fig:example:mi3:fig1})\\
    引用子圖片(b) (由於這張圖的caption是沒設定, 所以label無效)\\
    引用子圖片(c) (Fig \RefTo{fig:example:mi3:fig3})\\
    引用子圖片(d) (Fig \RefTo{fig:example:mi3:fig4})
  } % End of \item{}

  %
  \newpage
  \item
  {
    插入8張圖片, 2張圖一列, 只有主圖填了標題.\\
    \begin{verbatim}
    \InsertFigures
    [perrow = 2,
      caption = {8 figures and 2 figures per row}] %
    {[caption = { }]{./figure.png}}%
    {[caption = { }]{./figure.png}}%
    {[caption = { }]{./figure.png}}%
    {[caption = { }]{./figure.png}}%
    {[caption = { }]{./figure.png}}%
    {[caption = { }]{./figure.png}}%
    {[caption = { }]{./figure.png}}%
    {[caption = { }]{./figure.png}}
    \end{verbatim}

    \InsertFigures
    [perrow = 2,
      caption = {8 figures and 2 figures per row}] %
    {[caption = { }]{./example/how-to/write/figure/pic/CC-BY.png}}%
    {[caption = { }]{./example/how-to/write/figure/pic/CC-BY-NC.png}}%
    {[caption = { }]{./example/how-to/write/figure/pic/CC-BY-ND.png}}%
    {[caption = { }]{./example/how-to/write/figure/pic/CC-BY-SA.png}}%
    {[caption = { }]{./example/how-to/write/figure/pic/CC-BY.png}}%
    {[caption = { }]{./example/how-to/write/figure/pic/CC-BY-NC.png}}%
    {[caption = { }]{./example/how-to/write/figure/pic/CC-BY-ND.png}}%
    {[caption = { }]{./example/how-to/write/figure/pic/CC-BY-SA.png}}
  } % End of \item{}

  %
  \newpage
  \item
  {
    插入8張圖片, 3張圖一列, 只有主圖填了標題.\\
    \begin{verbatim}
    \InsertFigures
    [perrow = 3,
      caption = {8 figures and 3 figures per row}] %
    {[caption = { }]{./figure.png}}%
    {[caption = { }]{./figure.png}}%
    {[caption = { }]{./figure.png}}%
    {[caption = { }]{./figure.png}}%
    {[caption = { }]{./figure.png}}%
    {[caption = { }]{./figure.png}}%
    {[caption = { }]{./figure.png}}%
    {[caption = { }]{./figure.png}}
    \end{verbatim}

    \InsertFigures
    [perrow = 3,
      caption = {8 figures and 3 figures per row}] %
    {[caption = { }]{./example/how-to/write/figure/pic/CC-BY.png}}%
    {[caption = { }]{./example/how-to/write/figure/pic/CC-BY.png}}%
    {[caption = { }]{./example/how-to/write/figure/pic/CC-BY.png}}%
    {[caption = { }]{./example/how-to/write/figure/pic/CC-BY.png}}%
    {[caption = { }]{./example/how-to/write/figure/pic/CC-BY.png}}%
    {[caption = { }]{./example/how-to/write/figure/pic/CC-BY.png}}%
    {[caption = { }]{./example/how-to/write/figure/pic/CC-BY.png}}%
    {[caption = { }]{./example/how-to/write/figure/pic/CC-BY.png}}
  } % End of \item{}

  %
  \newpage
  \item
  {
    插入8張圖片, 4張圖一列, 只有主圖填了標題.\\
    \begin{verbatim}
    \InsertFigures
    [perrow = 4,
      caption = {8 figures and 4 figures per row}] %
    {[caption = { }]{./figure.png}}%
    {[caption = { }]{./figure.png}}%
    {[caption = { }]{./figure.png}}%
    {[caption = { }]{./figure.png}}%
    {[caption = { }]{./figure.png}}%
    {[caption = { }]{./figure.png}}%
    {[caption = { }]{./figure.png}}%
    {[caption = { }]{./figure.png}}
    \end{verbatim}

    \InsertFigures
    [perrow = 4,
      caption = {8 figures and 4 figures per row}] %
    {[caption = { }]{./example/how-to/write/figure/pic/CC-BY.png}}%
    {[caption = { }]{./example/how-to/write/figure/pic/CC-BY-NC.png}}%
    {[caption = { }]{./example/how-to/write/figure/pic/CC-BY-ND.png}}%
    {[caption = { }]{./example/how-to/write/figure/pic/CC-BY-SA.png}}%
    {[caption = { }]{./example/how-to/write/figure/pic/CC-BY.png}}%
    {[caption = { }]{./example/how-to/write/figure/pic/CC-BY-NC.png}}%
    {[caption = { }]{./example/how-to/write/figure/pic/CC-BY-ND.png}}%
    {[caption = { }]{./example/how-to/write/figure/pic/CC-BY-SA.png}}
  } % End of \item{}
  \end{enumerate}
% ------------------------------------------------
\EndChapter
% ------------------------------------------------

\newpage% ------------------------------------------------
\StartSection{表格 Table}{chapter:how-to:write:table}
% ------------------------------------------------

表格(Table)在任何情況下都是一個常用的顯示方式, 所以如何設計它都會有大量的玩法. 在正常Mircosoft Word這種有畫面的情況下, 可以慢慢拉出一個比較適合自己的, 但是在LaTex中這個過程會是十分的痛苦, 因為你沒法馬上知道修改後的畫面, 故要不斷測試才知道效果, 這樣會大大減低選用table的使用次數.

在一般任何的LaTex教學上, 如何編寫一個table出來都會是其中一項, 了解任何一個部份的寫法, 位置, 設定等. 但是由於那些資料十分的巨量 (不同寫法有不同效果), 所以這絕對不是使用本模版的大家想知道的東西, 故本模版不使用過往的方式, 而且直接教大家怎樣使用現有的online tool去處理掉這個問題.

以下的說明都是針對LaTeX Table Generator\RefBib{web:latex:table-generator}來進行說明. LaTeX Table Generator (Fig \RefTo{fig:how-to:table:table-generator})的頁面非常明瞭和簡單, 只要有過Mircosoft Word中的table設計的經驗, 應該要上手這個東西絕對不會很難.

\InsertFigure
  [scale=0.30,
  caption={LaTeX Table Generator頁面},
  label={fig:how-to:table:table-generator}]
  {./example/how-to/write/table/pic/table-generator.png}

% ------------------------------------------------
\newpage
\StartSubSection{產生LaTex}

  我們使用這工具就是要去產生LaTex用在論文當中, 所以這一步比其他的知識更為重要. 記得使用以下的步驟:

  \begin{enumerate}
  \item
  {
    使用畫面來設計table.
    \InsertFigure
      {./example/how-to/write/table/pic/table-view.png}
  } % End of \item{}

  \item
  {
    按Generate去產生LaTex.
    \InsertFigure
      {./example/how-to/write/table/pic/generate.png}
  } % End of \item{}

  \item
  {
    複製LaTex放到論文的".tex"中.
    \InsertFigure
      {./example/how-to/write/table/pic/latex-code.png}
  } % End of \item{}

  \item
  {
    執行XeLaTeX去產生效果.
  } % End of \item{}
  \end{enumerate}

  第1$\sim$3步會在整個設計table中常常都會使用, 所以會熟能生巧的. 而有經驗的人都知道, 第1步是最需要時間, 而第2$\sim$4步不用幾分鐘就能做完了, 所以只要用心的話, 多漂亮的table都是能弄出來的.

% ------------------------------------------------
\newpage
\StartSubSection{功能}

要設計一個複雜的table就需要足夠的功能才能慢慢弄, 所以在這邊介紹一些算是非常有用的功能.

\StartSubSection{File}

  在"File"中有幾個很有用的功能.
  \InsertFigure
    {./example/how-to/write/table/pic/menu-file.png}

  \begin{enumerate}

  % ------------------------------------------------
  \item
  {
    Import CSV file

    你可以直接upload一個CSV format的檔案之後弄table的外觀.
    \InsertFigure
      [scale=0.7]
      {./example/how-to/write/table/pic/csv.png}
  } % End of \item{}

  % ------------------------------------------------
  \newpage
  \item
  {
    Paste table data

    可以把Microsoft Excel的table直接做Copy \& Paste到這一邊來.
    \InsertFigure
      [scale=0.45]
      {./example/how-to/write/table/pic/paste.png}

    或是可以直接輸入資料來建立, 但要注意的是它只能接受CSV的寫法, 即是每一筆資料都是以","來分隔. 所以如果使用Fig \RefTo{fig:csv:enter-example-data}的寫法的話:
    \InsertFigure
      [scale=0.65,
        caption={Enter example data},
        label={fig:csv:enter-example-data}]
      {./example/how-to/write/table/pic/paste-data.png}

    會出現Fig \RefTo{fig:csv:result-example-data}的效果:
    \InsertFigure
      [scale=0.65,
        caption={Result of example data},
        label={fig:csv:result-example-data}]
      {./example/how-to/write/table/pic/paste-data-result.png}

  } % End of \item{}

  % ------------------------------------------------
  \newpage
  \item
  {
    Save table

    這online tool有一個十分有用的功能就是能把所做的table save下來, 只要輸入名字後再按download就會得到一個".tgn"檔案.
    \InsertFigure
      [scale=0.8]
      {./example/how-to/write/table/pic/save-table.png}

    \InsertFigure
      {./example/how-to/write/table/pic/save-tgn.png}

  } % End of \item{}

  % ------------------------------------------------
  \item
  {
    Load table

    在"Save table"中得到的".tgn"檔案就是使用這邊來重新讀取table.
    \InsertFigure
      [scale=0.8]
      {./example/how-to/write/table/pic/load-table.png}
  } % End of \item{}
  \end{enumerate}

\newpage
\StartSubSection{Edit}

  在"Edit"中有2個常用的功能

  \InsertFigure
    {./example/how-to/write/table/pic/menu-edit.png}

  \begin{enumerate}

  \item
  {
    Undo / Repeat

    很基本的重做上一步/下一步所做過的行為, 故不用解釋什麼.
  } % End of \item{}

  \item
  {
    Autosave

    這功能十分有用, 因為這tool是網頁tool, 所以正常重開網頁時會令到資料不見. 所以如果有把"Autosave"開啟的話, 那table就算接了"F5"都不會不見. (預設上應該會自動有開啟)
    \InsertFigure
      {./example/how-to/write/table/pic/edit-autosave.png}
  } % End of \item{}

  \end{enumerate}

% ------------------------------------------------
\newpage
\StartSubSection{Table}

  \begin{enumerate}

  \item
  {
    Set size

    這是table最基本的功能, 在Mircosoft Word時要插入多大的table時, 都要設定table的大小, 這邊正是那一個功能.
    \InsertFigure
      {./example/how-to/write/table/pic/table-set-size.png}
  } % End of \item{}

  \item
  {
    Clear table

    如果想把弄出來的table重新清掉所有設定和資料, 就是使用這一個.
    \InsertFigure
      {./example/how-to/write/table/pic/table-clear-table.png}
  } % End of \item{}

  \end{enumerate}

% ------------------------------------------------
\newpage
\StartSubSection{Extra options}

  在下方的"Extra options"有幾個基本的功能
  \InsertFigure
    [scale=0.5]
    {./example/how-to/write/table/pic/options.png}

\begin{enumerate}

  \item
  {
  Center table horizontally

  把整個table置中在頁面
  \InsertFigure
    [scale=0.5]
    {./example/how-to/write/table/pic/options-table-center.png}

  } % End of \item{}

  %\newpage
  \label{chapter:how-to:write:table:label-example}
  \item
  {
  Caption above / below, Label

  把圖表的標題要放在上方還是下方

  \InsertFigures
    [perrow = 2,
      caption = {Option of caption}] %
    {
      [scale=0.4,
      caption={標題放在上方}]
      {./example/how-to/write/table/pic/caption/above.png}
    }%
    {
      [scale=0.4,
      caption={標題放在下方}]
      {./example/how-to/write/table/pic/caption/below.png}
    }

  {\bf 注意:} 由於它沒有位置去修改caption和label, 所以要手動把caption和label中的內容修改.
  } % End of \item{}
\end{enumerate}

% ------------------------------------------------
\newpage
\StartSubSection{Style}

  在右邊可以設定table的style.
  \InsertFigure
    {./example/how-to/write/table/pic/style/style.png}

   正常在書本, 科學文章(如論文)和新聞中, table都是用三線式的方式, 因為這種的table簡單明瞭. 主要特點為整個table只有三條橫線, 上下兩端的線條較粗, 中間一條較細, 一般不使用分隔號.

  Fig \RefTo{table:style:sample-1}是一個例子分別是使用LaTex原版的顯示方式(Fig \RefTo{table:style:default-1})或是使用booktabs版的顯示方式(Fig \RefTo{table:style:booktabs-1}).

  \InsertFigures
    [perrow = 2,
      caption = {A sample between LaTex style and Booktabs style},
      label={table:style:sample-1}] %
    {
      [scale=0.3,
      caption={Default style},
      label={table:style:default-1}]
      {./example/how-to/write/table/pic/style/default-1.png}
    }%
    {
      [scale=0.2,
      caption={Booktabs style},
      label={table:style:booktabs-1}]
      {./example/how-to/write/table/pic/style/booktabs-1.png}
    }

  %\newpage
  而Fig \RefTo{table:style:sample-2}是2個版本都加上垂直線時候的樣子.

  \InsertFigures
    [perrow = 2,
      caption = {Table with horizontal line},
      label={table:style:sample-2}] %
    {
      [scale=0.3,
      caption={Default style}]
      {./example/how-to/write/table/pic/style/default-2.png}
    }%
    {
      [scale=0.25,
      caption={Booktabs style}]
      {./example/how-to/write/table/pic/style/booktabs-2.png}
    }

  就會發現booktabs版的中間的橫線比較細.

  這些都是一些細節問題, 如果想做簡單明瞭一些, 可以採用三線式表格, 但不是說只要是表格就必須使用三線式.

% ------------------------------------------------
%\newpage
\StartSubSection{其他}

  \begin{enumerate}

  \item
  {
    功能

    其他功能都很好理解的, 只要嘗試過就會明白, 所以不再作詳細解釋.
  } % End of \item{}

  \item
  {
    圖片

    這tool沒法插入圖片, 所以有關圖片的部份要自己加在table中, 請參考P. \RefPage{table:how-to:write:figure:insert-figure-into-table}, 但是在table中的figure是不能加標題和label.
  } % End of \item{}

  \item
  {
    備註

    而在產生出來的LaTex中, 可以看到這類的文字(Fig \RefTo{table:package:comment}). 在注解中所講的, 是指所產生出來的LaTex需要使用一些LaTex的工具, 但這些工具已被包在本模版中, 所以可以無視的.

    \InsertFigure
      [scale=0.7,
        caption={Package meno},
        label={table:package:comment}]
      {./example/how-to/write/table/pic/table-comment.png}
  } % End of \item{}
  \end{enumerate}

% ------------------------------------------------
\newpage
\StartSubSection{模版提供的功能}{subsection:how-to:write:table:api}

在畢業論文中, 表格的位置跟圖片一樣都是非常固定以中間為主, 而不一樣的東西主要是表格的標題位置和表格的設計, 同時為了幫同學們調整好表格的故使用斜線則必須自行在內容中進行修改位置, 大小和預設白色背景, 故本模版同時增加一個幫助你插入表格的功能.\\

  \begin{DescriptionFrame}
  \begin{verbatim}
  Content:   表格內容 (必填)
    只需要\begin{tabular} ... \end{tabular}這部份的內容

  Options 設定 (使用','來分隔, 不分先後順序)
    scale:   頁面的比例 (選填, 預設: 0.0)
    (0.0: 原大小; 1.0: 跟頁面一樣大;
     0.x: 以比例的大小; 個人推薦最大值為0.9, 因需保留小量左右的空白)
    caption: 標題 (選填)
    label:   標簽 (選填, 必須要配合caption使用, 否則無效)
    pos:   caption在表格的位置
      top為上方, bottom為下面 (選填, 預設: top)
    tabcolsep: 每一個表格左右的空白空間 (選填, 預設: 6pt)
    arraystretch: 每一個表格上下的空間 (選填, 預設: 1)
    opacity: 背景顏色透明度, 預設使用白色為背景 (選填, 預設: 0.75)
    (0.x ~ < 1.0: 透明; => 1.0: 不透明)

  插入表格
  \InsertTable[Options]{Content}

  E.g
    \InsertTable
    [caption={這 是 標 題}]
      {
        \begin{tabular}{ ... }
        ...
        \end{tabular}
      }
    \InsertTable
      [scale=0.5,
        pos=bottom,
        caption={這 是 標 題},
        label={this:is:label}]
      {
        \begin{tabular}{ ... }
        ...
        \end{tabular}
      }
  \end{verbatim}
  \end{DescriptionFrame}

% ------------------------------------------------

  \newpage
  {\bf 效果:}
  \begin{enumerate}

% ------------------------------------------------

  \item
  {
    標題在表格上方.
    \begin{verbatim}
    \InsertTable
      [caption={標題在上方}]
      {
        \begin{tabular}{|c|c|c|}
        \hline
         & Col 1 & Col 2 \\ \hline
        Row 1 & Value 1-1 & Value 1-2 \\ \hline
        Row 2 & Value 2-1 & Value 2-2 \\ \hline
        \end{tabular}
      }
    \end{verbatim}

    \InsertTable
      [caption={標題在上方}]
      {
        \begin{tabular}{|c|c|c|}
        \hline
         & Col 1 & Col 2 \\ \hline
        Row 1 & Value 1-1 & Value 1-2 \\ \hline
        Row 2 & Value 2-1 & Value 2-2 \\ \hline
        \end{tabular}
      }
  } % End of \item{}

% ------------------------------------------------

  \newpage
  \item
  {
    標題在表格下面.
    \begin{verbatim}
    \InsertTable
      [caption={標題在下面},
        pos=bottom]
      {
        \begin{tabular}{|c|c|c|}
        \hline
         & Col 1 & Col 2 \\ \hline
        Row 1 & Value 1-1 & Value 1-2 \\ \hline
        Row 2 & Value 2-1 & Value 2-2 \\ \hline
        \end{tabular}
      }
    \end{verbatim}

    \InsertTable
      [caption={標題在下面},
        pos=bottom]
      {
        \begin{tabular}{|c|c|c|}
        \hline
         & Col 1 & Col 2 \\ \hline
        Row 1 & Value 1-1 & Value 1-2 \\ \hline
        Row 2 & Value 2-1 & Value 2-2 \\ \hline
        \end{tabular}
      }
  } % End of \item{}

% ------------------------------------------------

  \newpage
  \item
  {
    Scale是用來調整表格的大小, 一般來講都不需要使用到這設定, 只有在特殊情況, 例如表格內容過多影響到寬度. 不同在Mircosoft Word中, 在LaTex中表格是會無視寬度是否超過頁面的, 故這就需要靠scale來調整.\\

Table \RefTo{table:how-to-write:table-example1} 是一個寬度超過頁面的例子, 而Table \RefTo{table:how-to-write:table-example2} 是把寬度控制跟頁面一樣闊, 但這就會沒有左右的空白空間, 而Table \RefTo{table:how-to-write:table-example3} 則是保留了左右的空白空間 (個人推薦最大值為0.9).

  \InsertTable
    [caption={表格寬度超過頁面},
      label={table:how-to-write:table-example1}]
    {
      \begin{tabular}{|c|c|c|c|c|c|c|c|c|c|c|c|c|c|c|c|}
      \hline
       & Col 1 & Col 2 & Col 3 & Col 4 & Col 5 & Col 6 & Col 7 & Col 8 & Col 9 & Col 10 & Col 11 & Col 12 & Col 13 & Col 14 \\ \hline
      Row 1 & Value & Value & Value & Value & Value & Value & Value & Value & Value & Value & Value & Value & Value & Value \\ \hline
      Row 2 & Value & Value & Value & Value & Value & Value & Value & Value & Value & Value & Value & Value & Value & Value \\ \hline
      Row 3 & Value & Value & Value & Value & Value & Value & Value & Value & Value & Value & Value & Value & Value & Value \\ \hline
      Row 4 & Value & Value & Value & Value & Value & Value & Value & Value & Value & Value & Value & Value & Value & Value \\ \hline
      \end{tabular}
    }

  \InsertTable
    [scale=1.0,
      caption={表格寬度設定scale=1.0},
      label={table:how-to-write:table-example2}]
    {
      \begin{tabular}{|c|c|c|c|c|c|c|c|c|c|c|c|c|c|c|c|}
      \hline
       & Col 1 & Col 2 & Col 3 & Col 4 & Col 5 & Col 6 & Col 7 & Col 8 & Col 9 & Col 10 & Col 11 & Col 12 & Col 13 & Col 14 \\ \hline
      Row 1 & Value & Value & Value & Value & Value & Value & Value & Value & Value & Value & Value & Value & Value & Value \\ \hline
      Row 2 & Value & Value & Value & Value & Value & Value & Value & Value & Value & Value & Value & Value & Value & Value \\ \hline
      Row 3 & Value & Value & Value & Value & Value & Value & Value & Value & Value & Value & Value & Value & Value & Value \\ \hline
      Row 4 & Value & Value & Value & Value & Value & Value & Value & Value & Value & Value & Value & Value & Value & Value \\ \hline
      \end{tabular}
    }

  \InsertTable
    [scale=0.9,
      caption={表格寬度設定scale=0.9},
      label={table:how-to-write:table-example3}]
    {
      \begin{tabular}{|c|c|c|c|c|c|c|c|c|c|c|c|c|c|c|c|}
      \hline
       & Col 1 & Col 2 & Col 3 & Col 4 & Col 5 & Col 6 & Col 7 & Col 8 & Col 9 & Col 10 & Col 11 & Col 12 & Col 13 & Col 14 \\ \hline
      Row 1 & Value & Value & Value & Value & Value & Value & Value & Value & Value & Value & Value & Value & Value & Value \\ \hline
      Row 2 & Value & Value & Value & Value & Value & Value & Value & Value & Value & Value & Value & Value & Value & Value \\ \hline
      Row 3 & Value & Value & Value & Value & Value & Value & Value & Value & Value & Value & Value & Value & Value & Value \\ \hline
      Row 4 & Value & Value & Value & Value & Value & Value & Value & Value & Value & Value & Value & Value & Value & Value \\ \hline
      \end{tabular}
    }

雖然內容可以保留在頁面中, 但看得出內容的文字會變小, 故表格的內容不能放過多內容, 否則會縮得十分的小.
  } % End of \item{}

% ------------------------------------------------

  \newpage
  \item
  {
    相反, 如果表格內容較少, 卻使用scale的話則會造成放大的行為.
Table \RefTo{table:how-to-write:table-example4} 是一個內容較少的表格, 而Table \RefTo{table:how-to-write:table-example5} 則設定了scale=0.9.

  \InsertTable
    [caption={內容較少的表格},
      label={table:how-to-write:table-example4}]
    {
      \begin{tabular}{|c|c|c|c|c|}
      \hline
       & Col 1 & Col 2 & Col 3 & Col 4 \\ \hline
      Row 1 & Value 1-1 & Value 1-2 & Value 1-3 & Value 1-4 \\ \hline
      Row 2 & Value 2-1 & Value 2-2 & Value 2-3 & Value 2-4 \\ \hline
      Row 3 & Value 3-1 & Value 3-2 & Value 3-3 & Value 3-4 \\ \hline
      Row 4 & Value 4-1 & Value 4-2 & Value 4-3 & Value 4-4 \\ \hline
      \end{tabular}
    }

% ------------------------------------------------

  \InsertTable
    [scale=0.9,
      caption={內容較少的表格, 但設定了scale=0.9},
      label={table:how-to-write:table-example5}]
    {
      \begin{tabular}{|c|c|c|c|c|}
      \hline
       & Col 1 & Col 2 & Col 3 & Col 4 \\ \hline
      Row 1 & Value 1-1 & Value 1-2 & Value 1-3 & Value 1-4 \\ \hline
      Row 2 & Value 2-1 & Value 2-2 & Value 2-3 & Value 2-4 \\ \hline
      Row 3 & Value 3-1 & Value 3-2 & Value 3-3 & Value 3-4 \\ \hline
      Row 4 & Value 4-1 & Value 4-2 & Value 4-3 & Value 4-4 \\ \hline
      \end{tabular}
    }
  } % End of \item{}

% ------------------------------------------------
  \newpage
  \item
  {
    使用透明度以能看到頁面中的學校浮水印. 相對於Figure來講, Table使用透明度是十分明顯的.\\

    \vspace{1.5cm}

    \InsertTable
      [caption={opacity使用預設}]
      {
        \begin{tabular}{llll}
        \hline
        Engine &  &  & OPEL Astra C16SE \\ \hline
        Displacement (cc) &  &  & 1598 \\
        Bore x stroke(mm x mm) &  &  & 79 x 81.5 \\
        Value mechanism &  &  & SOHC \\
        Number of valves &  &  & Intake 4, exhaust 4 \\
        Compression ratio &  &  & 9.8:1 \\
        Torque &  &  & 135/3400 Nm/rpm \\
        Power &  &  & 74/5800 kW/rpm \\
        Ignition sequence &  &  & 1-3-4-2 \\
        Spark plug &  &  & BPR6ES \\
        Fuel &  &  & 95 unleaded gasoline \\
        Cylinder arrangment &  &  & In-line 4 cylinders \\ \hline
        \end{tabular}
      } % End of  \InsertTable{}

      \InsertTable
        [caption={opacity使用0.4},
          opacity=0.4]
        {
          \begin{tabular}{|c|c|c|c|c|c|c|c|c|c|c|c|c|c|c|c|}
          \hline
           & Col 1 & Col 2 & Col 3 & Col 4 & Col 5 & Col 6 & Col 7 & Col 8 & Col 9 & Col 10 & Col 11 & Col 12 & Col 13 & Col 14 \\ \hline
          Row 1 & Value & Value & Value & Value & Value & Value & Value & Value & Value & Value & Value & Value & Value & Value \\ \hline
          Row 2 & Value & Value & Value & Value & Value & Value & Value & Value & Value & Value & Value & Value & Value & Value \\ \hline
          Row 3 & Value & Value & Value & Value & Value & Value & Value & Value & Value & Value & Value & Value & Value & Value \\ \hline
          Row 4 & Value & Value & Value & Value & Value & Value & Value & Value & Value & Value & Value & Value & Value & Value \\ \hline
          \end{tabular}
      } % End of  \InsertTable{}

  } % End of \item{}
% ------------------------------------------------

  \newpage
  \item
  {
    有時候在寫Pseudocode時會使用Pseudocode (Chap. \RefTo{chapter:how-to:write:pseudocode})外, 都可能會直接使用Table來顯示, 以下是使用Hello World為例子.

    \begin{verbatim}
      \InsertTable
        [caption={Hello World in C}]
        {
          \begin{tabular}{ll}
          \hline
          1. & \#include \textless stdio.h\textgreater \\
          2. &  \\
          3. & int main(void) \\
          4. & \{ \\
          5. & \ \ \ \ \ \ \ \ printf("hello, world"); \\
          6. & \} \\ \hline
          \end{tabular}
        }
    \end{verbatim}

  \InsertTable
    [caption={Hello World in C}]
    {
      \begin{tabular}{ll}
      \hline
      1. & \#include \textless stdio.h\textgreater \\
      2. &  \\
      3. & int main(void) \\
      4. & \{ \\
      5. & \ \ \ \ \ \ \ \ printf("hello, world"); \\
      6. & \} \\ \hline
      \end{tabular}
    }
  } % End of \item{}

相比Pseudocode的缺點是沒有自動算行數和Keyword沒有變粗體, 所有內容都由自己控制.

% ------------------------------------------------

  \newpage
  \item
  {
    使用tabcolsep來控制表格左右的空白空間

    \begin{verbatim}
      \InsertTable
        [tabcolsep = 18pt]
        {
          \begin{tabular}{|c|c|c|}
          \hline
          Title1 & Col 1 & Col 2 \\ \hline
          Row 1 & Value 1-1 & Value 1-2 \\ \hline
          Row 2 & Value 2-1 & Value 2-2 \\ \hline
          \end{tabular}
        }
    \end{verbatim}

    \InsertTable
      [tabcolsep = 18pt]
      {
        \begin{tabular}{|c|c|c|}
        \hline
        Title1 & Col 1 & Col 2 \\ \hline
        Row 1 & Value 1-1 & Value 1-2 \\ \hline
        Row 2 & Value 2-1 & Value 2-2 \\ \hline
        \end{tabular}
      }
  } % End of \item{}

  \newpage
  \item
  {
    使用arraystretch來控制表格上下的空間

    \begin{verbatim}
      \InsertTable
        [arraystretch = 2]
        {
          \begin{tabular}{|c|c|c|}
          \hline
          Title1 & Col 1 & Col 2 \\ \hline
          Row 1 & Value 1-1 & Value 1-2 \\ \hline
          Row 2 & Value 2-1 & Value 2-2 \\ \hline
          \end{tabular}
        }
    \end{verbatim}

    \InsertTable
      [arraystretch = 2]
      {
        \begin{tabular}{|c|c|c|}
        \hline
        Title1 & Col 1 & Col 2 \\ \hline
        Row 1 & Value 1-1 & Value 1-2 \\ \hline
        Row 2 & Value 2-1 & Value 2-2 \\ \hline
        \end{tabular}
      }
  } % End of \item{}

  \end{enumerate}

% ------------------------------------------------
\newpage
\StartSubSection{表格闊度和文字位置}

LaTeX Table Generator沒法設定每一個Column的闊度, 故本模版提供3個APIs來設定. 分別為:
  \begin{verbatim}
    L{ WIDTH }: 文字偏左
    C{ WIDTH }: 文字置中
    R{ WIDTH }: 文字偏右
  \end{verbatim}
這個東西是寫在`\verb|\begin{tabular}|'的位置, 例如可以寫\verb|\C{2.0cm}|, \verb|\L{20pt}|. 但比較推薦配合`\verb|\textwidth|'來使用, 因為是使用一行文字可使用的長度, 所以用來分成幾個column會比較好計算大約位置.

\begin{verbatim}
\InsertTable
{
  \begin{tabular}{C{0.2\textwidth} L{0.4\textwidth} R{0.35\textwidth}}
  \hline
  Title1 & Title2 & Title3 \\
  Center & Left & Right \\ \hline
  \end{tabular}
}
\end{verbatim}

    \InsertTable
      {
        \begin{tabular}{C{0.2\textwidth} L{0.4\textwidth} R{0.35\textwidth}}
        \hline
        Title1 & Title2 & Title3 \\
        Col1 & Col 2 & Col 3 \\ \hline
        \end{tabular}
      }

% ------------------------------------------------
\newpage
\StartSubSection{使用斜線}

斜線在表格上的設計是非常普遍, 但正如這一章開始時提到, LaTex在表格設計上不直覺, 有很多功能都要自行處理, 斜線這一功能正是其一. 在LaTeX Table Generator中是沒法弄出斜線的, 故需弄完表格後再修改內容. 以下的內容都是拿自斜線工具的文件 \RefBib{web:latex:diagbox-doc}, 只抽出一些重要內容.\\

  \begin{DescriptionFrame}
  \begin{verbatim}
  Options 斜線的設定 (使用','來分隔, 不分先後順序)
    width:  畫斜線的格子寬度 (選填, 推薦使用以cm/mm來設定)
    height: 畫斜線的格子高度 (選填, 推薦使用以cm/mm來設定)
    dir:   斜線的方向 (選填, 預設: NW)
      NW: 由左上向右下, NE: 由右上向左下
      SW: 由左下向右上, SE: 由右下向左上

  Content 表格在這格子中的內容文字 (可設2~3個)

  插入斜線
    \diagbox[Options]{Content}

  E.g
    \diagbox{A}{B}{C}

    \diagbox[dir=NW, width=1cm, height=1cm]{A}{B}
  \end{verbatim}
  \end{DescriptionFrame}

  一個最基本的例子:
  \begin{verbatim}
    \begin{tabular}{|l|ccc|}
      \hline
      \diagbox{Time}{Day} & Mon & Tue & Wed \\
      \hline
      Morning & used & used & \\
      Afternoon & & used & used \\
      \hline
    \end{tabular}
  \end{verbatim}

  \InsertTable
  {
    \begin{tabular}{|l|ccc|}
      \hline
      \diagbox{Time}{Day} & Mon & Tue & Wed \\
      \hline
      Morning & used & used & \\
      Afternoon & & used & used \\
      \hline
    \end{tabular}
  }

% ------------------------------------------------
\newpage

  如果是給3個的話:
  \begin{verbatim}
    \begin{tabular}{|l|ccc|}
    \hline
    \diagbox{Time}{Room}{Day} & Mon & Tue & Wed \\
    \hline
    Morning & used & used & \\
    Afternoon & & used & used \\
    \hline
    \end{tabular}
  \end{verbatim}

  \InsertTable
  {
    \begin{tabular}{|l|ccc|}
    \hline
    \diagbox{Time}{Room}{Day} & Mon & Tue & Wed \\
    \hline
    Morning & used & used & \\
    Afternoon & & used & used \\
    \hline
    \end{tabular}
  }
  % ------------------------------------------------
  如Column或Row標頭需要斷行的話都是可以:
  \begin{verbatim}
    \begin{tabular}{|c|}
    \hline
    \diagbox{Row\\header}{Col\\header} \\
    \hline
    \end{tabular}
  \end{verbatim}

  \InsertTable
  {
    \begin{tabular}{|c|}
    \hline
    \diagbox{Row\\header}{Col\\header} \\
    \hline
    \end{tabular}
  }

% ------------------------------------------------
\newpage

  使用以上的設定和組合可以玩出比較複雜的應用.

  \begin{verbatim}
    \begin{tabular}{|l|c|c|r|}
      \hline
      \diagbox{Time}{Day} & Mon & Tue & Wed\\
      \hline
      Morning & used & used & used\\
      \hline
      Afternoon & & used & \diagbox[dir=SW]{A}{B} \\
      \hline
    \end{tabular}
  \end{verbatim}

  \InsertTable
  {
    \begin{tabular}{|l|c|c|r|}
      \hline
      \diagbox{Time}{Day} & Mon & Tue & Wed\\
      \hline
      Morning & used & used & used\\
      \hline
      Afternoon & & used & \diagbox[dir=SW]{A}{B} \\
      \hline
    \end{tabular}
  }

% ------------------------------------------------
\newpage

最後就是斜線長度是跟隨表格中最寬的那個寬度, 故如果對寬度不滿意, 可自行調整\verb|\diagbox|的width.

  \begin{verbatim}
    \begin{tabular}{|c|} \hline
      \diagbox{A}{B} \\\hline
      Very long term \\\hline
    \end{tabular}
  \end{verbatim}

  \InsertTable
  {
    \begin{tabular}{|c|} \hline
      \diagbox{A}{B} \\\hline
      Very long term \\\hline
    \end{tabular}
  }

  調整成:
  \begin{verbatim}
    \begin{tabular}{|c|} \hline
      \diagbox[width=3cm]{A}{B} \\\hline
      Very long term \\\hline
    \end{tabular}
  \end{verbatim}

  \InsertTable
  {
    \begin{tabular}{|c|} \hline
      \diagbox[width=3cm]{A}{B} \\\hline
      Very long term \\\hline
    \end{tabular}
  }

% ------------------------------------------------
\EndChapter
% ------------------------------------------------

\newpage% ------------------------------------------------
\StartSection{公式 Equation}{chapter:how-to:write:equation}

% ------------------------------------------------
\StartSubSection{介紹}

公式(Equation)在都是一個常用的顯示方式, 雖然寫法都很固定, 但是內容可以十分豐富, 這產生大量的寫法. 而LaTex本身就擁有豐富的有關equation功能, Mircosoft Word都不一定有這麼多功能; 而且有一點Mircosoft Word是做不到, 但LaTex就很輕鬆的行為是: 你無法很簡單帶走你所寫的Equation, 拿去轉成圖片或是copy到另一個文件中.

但是在LaTex中, Equation跟Table(Chap \RefTo{chapter:how-to:write:table})都是一樣沒法即時知道修改後的畫面, 而且都會出現在基本教學中. 故本模版同樣教大家使用現有的online tool去處理掉這個問題.

% ------------------------------------------------
%\newpage
\StartSubSection{使用方式}
Equation有2種使用方式:
  \begin{enumerate}
    \item
    {
      跟文字寫在一起

      只要寫在2個\verb| $ |的符號之間, 即是\verb| $ ... $ |, 就可以顯示在文字之中.

      \noindent 例如:\\
      $E = mc^2$, 要寫成:\\
      \verb|      $E = mc^2$|\\
      而畢氏定理$c^2 = a^2 + b^2$, 要寫成:\\
      \verb|      畢氏定理($c^2 = a^2 + b^2$)是一個用來計算三角形的公式.|
    } % End of \item{}

    \newpage
    \item
    {
      使用本模版提供的語法.

      本模版結合了一些工具, 弄了\verb|\EquationBegin和\EquationEnd|這個語法, 在這個語法中所有公式都可以:
      \begin{enumerate}
        \item
        {
          可以在長公式的時候進行強制斷行

          只要在公式中插入\verb|\\|就可以強制斷行.
          \begin{verbatim}
            \EquationBegin
              x = a + b + c + \\
              d + e + f + g
            \EquationEnd
          \end{verbatim}

          {\bf 效果:}
          \EquationBegin
            x = a + b + c + \\
            d + e + f + g
          \EquationEnd
        } % End of \item{}

        %\newpage
        \label{chapter:how-to:write:equation:label-example}
        \item
        {
          在強制斷行下, 可以進行對齊位置

          使用\verb|&|就可以把你要的位置對齊, 以第一個\verb|&|為準則.
          \begin{verbatim}
            \EquationBegin
              x = &a + b + c + \\
              &d + e + f + g + \\
              &h + i + j + k
            \EquationEnd
          \end{verbatim}

          {\bf 效果:}
          \EquationBegin
            x = &a + b + c + \\
            &d + e + f + g + \\
            &h + i + j + k
          \EquationEnd
        } % End of \item{}

        \newpage
        \item
        {
          可設定標籤(Label)

          跟使用\verb|$...$|不一樣的是, 使用這個語法後, 每一個equation都會自動得到一個caption, 只要在\verb|\EquationBegin|加上\verb|{}|就可以為這個公式設定一個label來引用它.

          \begin{verbatim}
            \EquationBegin{eq:example:eq1}
              E = mc^2
            \EquationEnd
          \end{verbatim}

          e.g:
          \EquationBegin{eq:example:eq1}E = mc^2\EquationEnd

          使用\verb|\RefEquation|來引用:
          \begin{verbatim}Equation \RefEquation{eq:example:eq1}
              是Albert Einstein所想出來的.\end{verbatim}
          {\bf 效果:} Equation \RefEquation{eq:example:eq1} 是由Albert Einstein所想出來的.\\

          使用\verb|\RefEquationB|來引用 (數字會以\verb|'(X.X)'|)包起來:
          \begin{verbatim}這一條\RefEquationB{eq:example:eq1}
              是有名的物質轉成能量的equation.\end{verbatim}
          {\bf 效果:} 這一條\RefEquationB{eq:example:eq1}是有名的物質轉成能量的equation.
        } % End of \item{}
      \end{enumerate}
    } % End of \item{}
  \end{enumerate}

% ------------------------------------------------
\newpage
\StartSubSection{工具}

HostMath所提供的editor (Fig \RefTo{fig:how-to:equation:hostmath})\RefBib{web:latex:equation:hostmath}頁面簡單明瞭, 包含了所有LaTex支持的語法和斷行, 而且可以即時顯示LaTex語法和結果. 因為使用十分簡單, 所以本模版不作深入的介紹.

\InsertFigure
  [scale=0.31,
    caption={HostMath's latex equation editor},
    label={fig:how-to:equation:hostmath}]
  {./example/how-to/write/equation/pic/hostmath.png}

因為要修改內容, 但是每一個符號都有一個語法(而且顯示為藍色), 但是其實多到背不完, 所以根本不需要去記它們. 所以這個時候可以使用最簡單(笨蛋)的方式, 就是1對1來修改, 上面語法修改了什麼, 下面變了什麼, 那就代表那段語法代表什麼.

只要背3個重要的語法就能寫出你的equation:
  \begin{itemize}
    \item \verb|^|: 上標
    \item \verb|_|: 下標
    \item \verb|{ ... }|: 區域, 這一個區域的內容會放在同一個位置
  \end{itemize}

在Fig \RefTo{fig:how-to:equation:hostmath}已經舉了4個例子供大家理解.
% ------------------------------------------------
\newpage
\StartSubSection{轉成圖片}
HostMath是用來寫你的Equation, 但是如果你是把那條Equation轉成圖片的話, 可使用CodeCogs所提供的這個LaTex equation editor\RefBib{web:latex:equation:codecogs}.\\

這Editor (Fig \RefTo{fig:how-to:equation:codecogs})的頁面比HostMath來講有點簡陋, 但是重點是它可以轉出向量圖, 所以十分重要.

\InsertFigure
  [scale=0.27,
    caption={CodeCogs's latex equation editor},
    label={fig:how-to:equation:codecogs}]
  {./example/how-to/write/equation/pic/codecogs.png}

雖然簡陋, 但是使用上很簡單, 只要把Equation填進去, 之後選擇要ouput成什麼的圖檔, 那中間就會出現Equation的圖片和可按download的位置"Click here to Download Equation". 那download後就可以使用插入圖片 (Chap \RefTo{chapter:how-to:write:figure})的方式來插入用來當成論文的用圖片.

\newpage\input{./example/how-to/write/nomenclature/nomenclature}
\newpage% ------------------------------------------------
\StartSection{文獻引用 Bibliography/Reference}{chapter:how-to:write:bib}
% ------------------------------------------------

\StartSubSection{介紹}

Reference對論文來講十分重要的東西, 所以如果你引用的paper數量不少, 那在整理上會有點麻煩, 所以世界上有不少東西來管理這部份的資料, 如用的Word的話會配合Endnote.

而本模版是使用LaTex中的BibTex來管理, 你可以在`./content/references'找到3個`.bib'檔, 那正是你可以把你所引用的內容放在裡面.

Bib的分類滿多 (參考\RefBib{web:latex:bib_manage}), 但論文主要都是引用`book' (課本, 書籍等), `misc' (網頁, 任何其他東西), `inproceedings' (論文類)中的內容, 所以本模版提供的樣板檔案為`book.bib', `misc.bib' 跟 `paper.bib'.

\StartSubSection{使用方式}

任何放置論文的出版社(如ACM, IEEE, DBLP等), 都會為了方便別人去引用, 都會提供一些資料以給放在論文中引用. Fig \RefTo{fig:write:bib:1} 是以ACM Digital Library例子, 簡單說明如何使用BibTex來管理.

\InsertFigure
  [caption={ACM Digital Library例子},
    label={fig:write:bib:1}, scale=0.5]
  {./example/how-to/write/bib/pic/1.png}

\InsertFigure
  [caption={BibTex的位置},
    label={fig:write:bib:2}, scale=0.4]
  {./example/how-to/write/bib/pic/2.png}

在畫面右方會看到`Export Formats'的位置, 會看到如fig \RefTo{fig:write:bib:2}中一個的BibTex的按鈕.

\InsertFigure
  [caption={BibTex資料},
    label={fig:write:bib:3}, scale=0.5]
  {./example/how-to/write/bib/pic/3.png}

按它後就會出現如fig \RefTo{fig:write:bib:3}這個畫面, 這個就是要填進Bib的資料, 所以把這個東西複製到Bib檔內.

\InsertFigure
  [caption={整理/使用BibTex},
    label={fig:write:bib:4}, scale=0.5]
  {./example/how-to/write/bib/pic/4.png}

但複製完後要改一個東西, 第一行是所謂的label部份(參考Chap \RefTo{chapter:how-to:write:label}), 所以要改成一個自己能記得的label以方便在內容中來引用.

%有什麼問題可以去問Google\cite{website:google}老師. (如果有設定references用的檔案, 即使用了ReferencesFiles, 那必須至少要存在一個cite才不會顯示錯誤.)

% ------------------------------------------------
\EndChapter
% ------------------------------------------------

\newpage% ------------------------------------------------
\StartSection{虛擬程式碼(Pseudocode)}{chapter:how-to:write:pseudocode}
% ------------------------------------------------

Pseudocode在資訊類的paper是很常見, 雖然這東西冷門, 但是有它的存在意義.
如果不想用Pseudocode來寫, 可考慮使用Table來做 (考慮章節 \RefTo{subsection:how-to:write:table:api}).

而由於需要寫Pseudocode的人, 理論上都100\%會寫程式, 所以有關這邊會直接使用例子(基本的function, if-elseif-else, while, return, switch-case)來說明, 靠例子應該就能寫出你所要的Pseudocode.

唯一注意的是需要使用:\\
'\verb|\Statex|'來斷一行空行\\
'\verb|\State|'來斷一行以寫新code在後面

% ------------------------------------------------

\newpage
例子1:
\begin{algorithm}
  \caption{My algorithm (function A)}
  \label{algo:functionA}

  \begin{algorithmic}[1]
    \Function{function\_name\_a}{arg1, arg2}
      \If{conditionA}
        \State ...
      \ElsIf{conditionB}
        \State ...
      \Else
        \State ...
      \EndIf
      \Statex
      \If{condition1}
        \State ...
      \Else
        \If{condition2}
          \State ...
        \Else
          \State ...
        \EndIf
      \EndIf
      \Statex
      \For{condition}
        \State ...
      \EndFor
    \EndFunction
  \end{algorithmic}
\end{algorithm}

\newpage
針對function A (Algorithm \RefTo{algo:functionA}), 它的LaTex寫法為:\\

\begin{DescriptionFrame}
  \begin{verbatim}
\begin{algorithm}
  \caption{My algorithm (function A)}
  \label{algo:functionA}

  \begin{algorithmic}[1]
    \Function{function\_name\_a}{arg1, arg2}
      \If{conditionA}
        \State ...
      \ElsIf{conditionB}
        \State ...
      \Else
        \State ...
      \EndIf
      \Statex
      \If{condition1}
        \State ...
      \Else
        \If{condition2}
          \State ...
        \Else
          \State ...
        \EndIf
      \EndIf
      \Statex
      \For{condition}
        \State ...
      \EndFor
    \EndFunction
  \end{algorithmic}
\end{algorithm}
  \end{verbatim}
\end{DescriptionFrame}

% ------------------------------------------------

\newpage
例子2:
\begin{algorithm}
  \caption{My algorithm (function B)}
  \label{algo:functionB}

  \begin{algorithmic}[1]
    \Function{functionNameB}{}
      \State ...
      \State Some code here
      \State ...
      \Statex
      \While{condition3}
        \State ...
      \EndWhile
      \Statex
      \Repeat
        \State ...
      \Until{condition3}
      \Statex
      \Switch{condition4}
        \Case{condition5} ... \Break \EndCase
        \Statex
        \Case{condition6}
          \State ...
          \State \Break
        \EndCase
        \Statex
        \Default
          \State ...
        \EndDefault
      \EndSwitch

      \Statex\State \Return retValue
    \EndFunction
  \end{algorithmic}
\end{algorithm}

\newpage
針對function B (Algorithm \RefTo{algo:functionB}), 它的LaTex寫法為:\\

\begin{DescriptionFrame}
  \begin{verbatim}
\begin{algorithm}
  \caption{My algorithm (function B)}
  \label{algo:functionB}

  \begin{algorithmic}[1]
    \Function{functionNameB}{}
      \State ...
      \State Some code here
      \State ...
      \Statex
      \While{condition3}
        \State ...
      \EndWhile
      \Statex
      \Repeat
        \State ...
      \Until{condition3}
      \Statex
      \Switch{condition4}
        \Case{condition5} ... \Break \EndCase
        \Statex
        \Case{condition6}
          \State ...
          \State \Break
        \EndCase
        \Statex
        \Default
          \State ...
        \EndDefault
      \EndSwitch

      \Statex\State \Return retValue
    \EndFunction
  \end{algorithmic}
\end{algorithm}
  \end{verbatim}
\end{DescriptionFrame}


% ------------------------------------------------
\EndChapter
% ------------------------------------------------
